Although TraceR-CODES is powerful, it does not support all functions required by
our study.
%In particular, it does not support non-standard fat-trees (which are
%often deployed) and some message scheduling schemes that we intend to
%investigate. 
We added a new graph-based network model to CODES, which
gives us the flexibility to perform simulations on realistic fat-tree networks.
We also added message scheduling mechanisms that are used in this study.
Further, we modified the Tracer-CODES simulator so that it can track 
 communication-computation overlap in applications.

%To study the hardware parameters in this work,
%\begin{itemize}
%\item We added a new network model.
%\item We added various NIC-scheduling mechanisms in Tracer-CODES.
%\item We also modified
%  the
%Tracer-CODES simulator so that it can track the timings of
%communication, computation overlap in applications.
%\end{itemize}

\subsection{Simulating an arbitrary network topology}

TraceR-CODES provides many built-in interconnection networks, including fat-tree
and dragonfly. However, the connectivity within these built-in networks is
rigid, and may not exactly match networks deployed in production supercomputers.
Therefore, we implement a new network model, which we call the {\em arbitrary
graph model}, that allows users to explicitly define their network topology.
%This, with the arbitrary graph model, any network topology, regular or
%irregular, symmetrical or non-symmetrical can be simulated.
%This is a new feature that increases the capability of Tracer-CODES.

%from implementing standard, often
%symmetrical topologies, such as fat-tree, dragonfly, etc.
%to any arbitrary network model. This is important because in most cases,
%the interconnect topologies that are used in real computing facilities are
%a bit different from the built-in topologies in Tracer-CODES even when the
%network is of the same type. Arbitrary Graph Network Model is designed to
%take any network topology and feed it to the Tracer-CODES simulator. It
%takes the network description XML file and creates the network topology
%from it. Tracer-CODES simulator can then replay traces on this network model.

\subsection{Simulating message scheduling in the NIC}

The network interface controller (NIC) in contemporary HPC systems is
responsible for scheduling and packetizing messages.

\vspace{0.08in}
\noindent{\bf FCFS scheduling:} 
Messages are inserted
at the back of the scheduling queue, as and when they arrive.
During the packetization process, %the scheduler pops the elements from the head of
%the scheduling queue, creates the packet from that message, and pushes it back
%to the top.
the scheduler keeps creating packets from the top of the queue until
the entire message is packetized before it packetizes the next message in the queue. 
%This is what typically happens when there is
%only one virtual channel in a NIC/network.

\vspace{0.08in}
\noindent{\bf RR scheduling:} Messages are inserted into the scheduling queue of
the network interface, as and when they arrive. During the packetization process,
the scheduler creates one packet for a message and then moves to the next message:
all messages are considered in a round-robin manner. 
%However during the packetization process, the scheduler pops the top message from
%scheduling queue, creates a packet from it and then inserts to the tail. The scheduler
%then moves to the next message in the queue, which is currently at the top of the
%scheduling queue. The scheduler repeats the process until all the messages in the
%scheduling queue are packetized before it again creates a packet from the first message
%which entered the queue initially.
RR not only allows concurrent communication
progress for several communicating-pairs, but may also help the network in
better utilizing multiple communication paths. While desirable, such a scheme is
difficult to implement in the hardware as the number of concurrent messages can
be very large.

%default first come first serve (FCFS) message
%scheduling. 
\vspace{0.08in}
%In RR-N, the messages are entering the scheduling queue in the network interface
%in a first come first serve manner with the first N message in the queue, waiting
%for their turn of packetization. Each of those N messages is first packetized one
%after the other. When one of the N messages is completely packetized, the immediate
%next message in the queue takes its place. 

\vspace{0.08in}
\noindent{\bf RR-N scheduling:} In this scheme, $N$ is a parameter. RR-N is similar to RR,
except that instead of packetizing every message in the scheduling queue in
a round-robin manner,
the scheduler packetizes the top $N$ messages in the scheduling queue. For example,
in RR-2, the scheduler only packetizes the first 2 messages for communication.
This newly added scheme simulates the real world scenario where a limited
number of hardware queues are available at a NIC, which are used to keep
multiple message in-flight concurrently.
%and thus
%packetization of only these many messages can happen concurrently.  

\subsection{Simulating multi-GPU nodes in TraceR-CODES}
\label{sec:gpu_tracer}

In TraceR, computation is simulated by fast-forwarding the simulation time by a
desired amount. Thus, to emulate GPU execution in place of a CPU, we 
reduce the time spent in computational regions of an application
by a certain factor. For each application, this factor is based on the 
speed up observed by us on Sierra supercomputer in comparison to the CPU cluster
used to collect the application traces. Further, to simulate the impact of 
high-bandwidth links among the GPUs in a node (e.g. NV links), we added dedicated 
communication routes among the GPUs in a node.
