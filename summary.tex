This chapter explores the adaptation of software-defined networking (SDN) techniques to the distinct communication patterns of high performance computing (HPC) applications. While SDN has been successful in traditional data center and wide-area network environments, its use in HPC systems has remained limited due to challenges posed by fine-grained and phase-driven communication behavior. This study addresses those limitations by introducing novel flow classification and communication phase identification mechanisms specifically designed to align with HPC workloads.

The core of this work involves a detailed simulation-based evaluation using the TraceR-CODES framework. Three flow classification approaches are considered: threshold-based classification, deep neural network (DNN)-based classification, and user-input-based classification. These are applied in conjunction with three SDN routing strategies: greedy routing, optimal routing, and adaptive routing. The techniques are evaluated across real and synthetic HPC workloads using two common network topologies: a 1024-node full-bisection fat-tree and a 1536-node 3-to-1 tapered fat-tree.

The results show that the strategies implemented in SDN routing perform very well under single-path routing. They significantly reduce network congestion and ensure the efficient execution of HPC applications within SDN-enabled systems. The multipath variant, SDN-adaptive, dynamically selects between SDN and adaptive routing and achieves performance that is either better than or comparable to adaptive routing alone, depending on the type of application and the density of communication. Furthermore, the use of phase-aware routing and early flow identification, particularly through user input or DNN-based models, plays a key role in adapting routing behavior to dynamic traffic patterns and improving overall communication efficiency.

By aligning SDN’s programmability and global control with the communication characteristics of HPC workloads, this work demonstrates that SDN can serve as a robust and efficient networking solution for modern supercomputing platforms. These contributions support the broader integration of SDN in HPC systems and provide a solid foundation for future research in intelligent, adaptive network control.
