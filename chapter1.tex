\chapter{Introduction} 
In recent years, the demand for High-Performance
Computing (HPC) applications has surged, driven by the increasing complexity of
scientific computations and data-intensive tasks. Despite the remarkable
progress in HPC infrastructure, challenges persist in optimizing communication
performance, particularly in modern HPC environment like cloud data centers. 
This prospectus aims to address these challenges by
incorporating Software-Defined Networking (SDN)~\cite{kreutz2014software} enhancements for optimizing routing for these 
state of the art HPC environments, thereby improving the efficiency and reliability of communication
for HPC applications.  

Cloud data centers, such as those offered by GCP and
Azure, heavily rely on fat-tree topologies ~\cite{leiserson1985fat} for their network architecture ~\cite{de2022}.
However, the bottleneck in these systems often lies in communication efficiency,
hindering the seamless execution of HPC applications. 
Between 2010 and 2018, there was a remarkable 65-fold surge in 
the computational throughput of the Top 500 HPC systems ~\cite{t500}, whereas the increase 
in offnode communication bandwidth was comparatively modest, standing at only 4.8 times ~\cite{bergman2018empowering} ~\cite{michelogiannakis2019bandwidth}.
By enhancing the routing
techniques within fat-tree topologies through SDN, this research seeks to
alleviate these bottlenecks and facilitate the deployment of HPC workloads on
cloud platforms. Such improvements hold significant implications for scientific
computing, enabling researchers to harness the full potential of HPC resources
for groundbreaking discoveries across various fields.  

The research methodology
entails a comprehensive approach to maximize the performance of fat-tree
topologies with SDN enhancements. 


Initially, a thorough validation of the
simulation model against real-world HPC systems, such as Quartz ~\cite{quartz}, will be
conducted to understand the effects of network parameters (e.g., link bandwidth,
NIC scheduling) on communication performance. 


Subsequently, three distinct SDN routing strategies will be implemented and evaluated: greedy single-path
routing, multipath routing, and heuristic-based optimal routing. These
strategies will be tested within a full bisection fat-tree topology to assess
their efficacy in improving communication performance for HPC applications.  

The primary contribution of this research lies in its exploration of how network
parameters impact communication in next-generation HPC systems. By integrating
SDN enhancements into fat-tree topologies, novel insights will be gained into
optimizing communication efficiency for data-centric applications. Furthermore,
this work will extend existing knowledge by demonstrating the potential of
SDN-driven routing techniques to significantly enhance the performance of HPC
workloads in cloud environments.  

The structure of this prospectus aligns with
the outlined research objectives. Chapter 2 will provide a comprehensive
overview of interconnection technologies and SDN fundamentals. In Chapter 3, the
focus will shift to examining the influence of network parameters on modern HPC
systems. Chapter 4 will delve into the implementation of SDN enhancements within
HPC environments and evaluate their impact on application performance. Finally,
Chapter 5 will offer concluding remarks and insights derived from the research
findings.

