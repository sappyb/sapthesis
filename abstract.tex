The demand for High-Performance Computing (HPC) applications has surged due to
the increasing complexity of scientific computations. The introduction of
new GPU based compute nodes and SDN technology alters the balance between
computation and communication aspects within the system, shifting the
communication-to-computation ratio. As computation speeds up with the addition
of more powerful GPU-based compute nodes, communication fails to scale
proportionally. New technologies like Software-Defined Networking (SDN) attempt
to bridge this gap by providing improved resource management during
communication. Still these challenges prevails, which in turn necessitate thorough investigation, thus
prompting the need for research on how applications perform on machines equipped
with new technologies. This dissertation proposes leveraging SDN in conjunction
with GPU acceleration to enhance communication efficiency of applications in HPC
environments and also find the optimal system configuration for running
different applications in these environments. By alleviating communication
bottlenecks, this research aims to facilitate seamless execution of HPC
workloads, thereby enabling groundbreaking discoveries in scientific computing.
