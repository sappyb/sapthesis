The demand for High-Performance Computing (HPC) applications has surged due to
the increasing complexity of scientific computations. The introduction of
GPU-based compute nodes
alters the balance between computation and communication aspects within the
system, shifting the communication-to-computation ratio.
As computation speeds up with the addition of powerful GPU-based compute
nodes, communication fails to scale proportionally. New networking
technologies like Software-Defined Networking (SDN) attempt to alleviate
this problem by providing
improved resource management during communication.
%Still these challenges prevails, which in turn necessitate thorough
%investigation, thus
%prompting the need for research on how applications perform on
%machines equipped
%with new technologies.
This dissertation proposes to leverage SDN to enhance
communication efficiency in HPC environments with GPU nodes
and to find efficient system configurations for running
HPC applications in such environments. By alleviating communication
bottlenecks, this research aims to facilitate seamless execution of HPC
workloads, thereby enabling groundbreaking discoveries in scientific computing.
