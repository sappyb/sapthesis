
HPC applications exhibit phased behavior with alternating
computation and communication phases. Since communications in different
phases do not overlap, network resources such as communication channels
allocated to communications in one phase can be reused for communications
in another phase. Hence, identifying communication phases, which is unique
in the HPC environment, allows the SDN to manage resources more effectively.

\subsection{Phase Identification With User Hints}

Communication phases in an HPC application can be easily identified
inside a program: there are often a section of code corresponding
for the communications in the program. Figure~\ref{code.laghos.1}
shows the code snippet for Laghos with communication phase marked. 
Here, after initiating MPI\_Allreduce
communications, the node performs MPI\_Barrier to make sure all ranks 
have the data, before initiating the computation phase where
it calculates the density.
Now, the communication pattern may be different in the two
communication phases in the given code as the comm world is calculated
at runtime. Being able to reuse network resources in different phases will
significantly improve network resource utilization. 
I propose to have an API for HPC application to give
information about phases. With the API, the user can insert a marker before
and after each communication
phase in the program to inform the network the starting and
ending of a communication phase. With the assistance from the user,
the SDN network can detect communication phases in an HPC application
without minor overheads: once all processes for an application
enter a computation phase, the application is in the computation phase and
resources for communications in the previous phase can be released; as
soon as one process enters a communication phase, the application enters
the communication phase. 

\begin{figure}[H]
\begin{lstlisting}[breaklines, language=C++, frame=single, tabsize=4, basicstyle=\ttfamily]
LagrangianHydroOperator (...){
        ...
        ParMesh *pm = H1FESpace.GetParMesh();
        MPI_Allreduce(&loc_area, &glob_area, 1, MPI_DOUBLE, MPI_SUM, pm->GetComm
());
        ...
}
\end{lstlisting}
\caption{Laghos code snippet}
\label{code.laghos.1}
\end{figure}

\begin{comment}
Communications in different phases do not happen at the same time. As such, network
resources (such as communication channels) can be allocated
independently in different phases.
Hence, accurately identifying communication phases can improve
the SDN support for HPC applications. Traditionally SDN schemes such
as the Hedera's threshold-based scheme automatically detect communication
changes when the flow statistics are collected and processed. However,
the granularity of such detection is the same as that for flow statistics
collection and processing, and network reconfiguration,
which is in the order in seconds. Such a coarse granularity is ineffective
for tightly coupled HPC applications whose phases can change in
subseconds or even submilliseconds. 
\end{comment}

%In this section I will talk about the phases of 
%HPC application and how to detect them during application runtime.

\subsection{Dynamic Communication Phase Identification}

\begin{comment}
 Flow information used to
classify elephant flows in current phase should ignore the flow information for the flows in the
previous phase as these two set of flows  don't occur together.
In order to
determine if the network is in a communication phase, the network must be
probed at intervals smaller than the polling interval to check if data is being
sent above a certain threshold. If it is, then the network is in a
communication phase; otherwise, it is mostly in a computation phase.
If the
network transitions from a computation phase to a communication phase or vice versa during
probing, previous flow paths for elephant flows are disregarded, and the flow
informations for each flow in the current phase is considered for
prediction. 
\end{comment}

Without the hints from application, communication phases in an HPC
application can be detected by finding the computation period in the
application when very few communications are performed. 
My dynamic communication phase identification algorithm
is shown in Figure~\ref{alg:phase_detection}. I first decide the interval
value for a minimal computation phase (e.g. 100ms). The phase detection
algorithm is executed periodically at the interval boundary to determine
whether phase is a communication phase or a computation phase depending
whether the total data communicated in the application during the phase
passes a threshold value. If a current phase is a computation phase and
the previous phase is a communication phase, network resources allocated
for the application in previous phases are released. 


\begin{algorithm}
\DontPrintSemicolon

\caption{Dynamic phase identification algorithm}
\label{alg:phase_detection}

Let Total\_data be all data sent by the application in the current interval\;
    \If{($Total\_data > Threshold$)}{
        Current phase is a communication phase\;
    }
    \Else{
      Current phase is a computation phase\;
      \If {(The previous phase is a communication phase)} {
        Release resources allocated in previous phases
        }
    }

\end{algorithm}


