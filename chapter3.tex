\chapter{Design and Evaluation of Techniques for HPC platforms with SDN-capable Interconnects} 

Software-defined networking (SDN)~\cite{kreutz2014software} has shown great
promise and has been widely deployed in data centers, campus networks,
and wide-area networks. SDN has the ability to manage traffic at the flow
level using the logically centralized global network view and to optimize
the network resource utilization for global optimality, which may
significantly improve network performance over the traditional networking
infrastructure~\cite{tr2016sdn}.

Although SDN features are also attractive to high performance
computing (HPC) systems and applications, SDN has yet to be widely adopted
in the HPC domain. Existing SDN techniques optimize for Internet
and data-parallel applications (e.g. Hadoop and map-reduce
applications)~\cite{he2016firebird}. The communication characteristics of
HPC applications are different from those of Internet and data-parallel
applications. For example, many HPC applications simulate
physical processes over numerous time steps,
with each time step performing similar tasks: the execution of
such applications exhibits phased behavior with alternating computation
and communication phases. During the computation phases, few communications
are performed; and the communications often repeat themselves in
different time steps. Additionally,
communications in HPC applications are often static in that they are known
to the application developers or can be analyzed statically or dynamically
~\cite{faraj2002communication,hong2013achieving}. Exploring such features
in HPC applications and systems will allow SDN to support
communications more effectively and to perform its tasks more efficiently.

In this work, I develop techniques to adapt SDN to HPC workloads and systems,
taking HPC application characteristics into account. The techniques
include flow identification, phase identification, and flow scheduling.
Flow identification identifies the types of flows
in applications, which is essential
for  an SDN-capable network to achieve high performance.
Phase identification identifies communication phases in applications, which
allows network resources to be utilized more effectively. Flow scheduling
schedules the communication to achieve target optimization objectives.
To maximize the effectiveness, my techniques treat
static and dynamic communications in HPC applications differently.
Dynamic communications are handled using techniques similar to those
in the traditional SDN networks. For static communications, I propose
to enhance SDN with an API for HPC applications
to give hints to the SDN system (e.g. whether a flow is an elephant flow
or will likely be an elephant flow). For such communications,
the network system relies on the upper layer to obtain communication
information. This is similar to the intent-based API \cite{Coflow2012}
where application developer and the network system work
together for flow and phase identification.

I conducted extensive simulation experiments using the
TraceR-CODES
\cite{jain2016evaluating,mubarak2016enabling,jain2017predicting}
PDES ~\cite{fujimoto1990parallel} simulator on a 3-level
fat-tree topology ~\cite{leiserson1985fat} ~\cite{al2008scalable}. My
simulation results reveal that my techniques improve the performance
over the existing SDN
scheme for applications with both static and dynamic communications.
%For example, for the static Stencil4d benchmark with the near neighbor
%communications, my technique improves the performance by around XXXXXX\%.
%For the dynamic Milc benchmark where multiple patterns are
%performed in different iterations, my technique improves over
%existing SDN scheme by XXXXXX\% on average across all three
%fat-tree topologies.
The main contributions of this work include the following:
%
\begin{itemize}
\item I identify the features in HPC applications that can be used
  to enhance the effectiveness of SDN.
\item I develop techniques to adapt SDN to HPC applications and
  systems by exploiting the identified HPC features.
\item I perform extensive simulation to evaluate and validate
  the proposed techniques.
\end{itemize}


\section{Flow Identification in HPC Environments}
\label{sec:flow_identification}
Data centers and HPC systems differ significantly in terms of their traffic characteristics. While the majority of traffic in a data center is small-scale, unpredictable, and not localized to any one level of the switch, traffic in an HPC application is primarily near-neighbor traffic, and may not be modest flows. Traditional flow identification methods may not be effective
in HPC environments. To effectively support SDN in the HPC environment,
novel techniques that take HPC communication characteristics into
consideration are proposed. Since a significant portion of
communications in HPC applications are static, we propose to have an API for
applications to provide flow information to the network directly. For
dynamic communications, we develop a machine learning based approach for
flow identification.

Figure~\ref{fig:flow_schemes} shows the high-level view of the proposed
flow classification systems. The flow classification techniques can
be classified into two types, those {\em with no extra user information} and
those {\em with user information}. The components below SDN API are similar to
traditional SDN systems. The flow classifier may take traffic statistics from
SDN switches and performs flow classification with no extra user information.
The classifier also allows HPC applications (SDN user) to directly give
hints about their communications to the network through an API
(similar to the intent-based API \cite{Coflow2012}), and classifies
such flows based on user information.

\begin{figure}[h]
\centering
    \includegraphics[width=0.8\columnwidth]{figs/Fig3.png}
\caption{High-level view of our flow classification schemes}
  \label{fig:flow_schemes}
\end{figure}

\subsection{Flow identification without user input}


\vspace{0.08in}
\noindent{\bf Threshold-based scheme}:
Existing flow classification methods including end host-based
management~\cite{xu2015identifying},
packet sampling~\cite{suh2014opensample}~\cite{afek2015sampling},
and polling per-flow statistics~\cite{yang2020flow}, do not assume
the type of applications. In this work, we compare our proposed techniques with
the classic elephant flow detection using a polling-based approach,
which is based on Hedera's architecture \cite{al2010hedera}.
Hedera's control loop comprises
three main components: flow detection, channel calculation, and channel
placement. Initially, significant 
flows are detected at the edge switches, and appropriate channels for these 
large flows are calculated using placement algorithms, taking into account 
their natural demand. These pathways are then placed on the switches.
%Hedera's architecture supports all common multi-rooted tree topologies.

Due to the overhead concerns, there is a limit how fast the flow statistics
are gathered, and path calculation and installations are performed, which is typically
in the order of seconds \cite{al2010hedera}.
However, in the HPC environment, depending
on applications, the time for each iteration may be much less than a second.
Hence, these existing flow classification schemes will miss many iterations
of application execution and not effective for such HPC applications.
To perform flow classification effectively for HPC systems, we develop
a machine learning based approach using deep neural network (DNN)
for flow classification.

\vspace{0.08in}
\noindent{\bf Deep learning-based scheme}:
As mentioned earlier, communications in HPC applications exhibits phase
behavior. If such behavior can be characterized and learned, we can classify
the flows quicker (after a small number of phases).
The information to differentiate between large elephant flows from
small mice flows is fundamentally captured by the time sequence of packets.
By training a Deep Neural Network (DNN) model using the time sequence of
packets from HPC workloads, the DNN model is able to recognize the
patterns of elephant flows in HPC applications. We note that the patterns
learned by the DNN model goes beyond simple statistics like
the existing threshold-based scheme: the model can also reflect more
sophisticated patterns such as the phased behavior. 


\vspace{0.08in}
\noindent{\em Model architecture}:
Our DNN model consists of an input layer of size 300 to accept the input data
which consists of data sent across various intervals of time and a single dense layer with
one output neuron, which uses the sigmoid activation function to classify the flows as elephants or mice.
The sigmoid function is commonly used for binary classification tasks
as it outputs a value between 0 and 1, which can be 
interpreted as the probability of the input belonging to the positive class. 
The model uses the Nadam optimizer with a learning rate of 0.01. Nadam is an 
extension of the popular Adam optimizer that incorporates Nesterov momentum, 
which helps to accelerate convergence. The model is compiled with binary 
cross-entropy loss, which is a standard loss function used for binary 
classification tasks.

\vspace{0.08in}
\noindent{\em Data collection and model training}:
We utilized the TraceR-CODES simulator to execute representative HPC 
workloads, namely Random-Permutation, Shift-256, Stencil3d, Stencil4d, Milc, Nekbone, Subcom3d-a2a, Kripke ~\cite{kripke}, Laghos, 
SW4lite ~\cite{sjogreen2018sw4}, and AMG ~\cite{amg}, 
for ranks ranging from 32 to 512. More detailed description of these
workloads is given in Section~\ref{sec:exp}.
Flow statistics were collected every 0.3 
seconds or 1\%
of the simulation to maintain minimal granularity. The flow data 
was gathered independently for each rank and merged to train the DNN 
model. For training purposes, the flows are marked as elephant or
mouse flows with a cut-off of $3 * 10^8$ bytes of data transfer in 90 seconds: 
Elephant flows were those transferring more than 300 MB of data in
90 seconds, while mouse flows were 
those transferring less. We employed a min-max scalar to scale the flow data 
before inputting it into the dense neural network layer for forecasting. The 
developed model was used offline to forecast network flows of elephants
or mice for systems running a combination of the aforementioned
applications.

During training, the model is trained on the input data and
corresponding binary labels. The total data is split as follows 20\%
of the training data is used for validation, while the remaining 80\% is 
used for training. Table~\ref{tab:model_acc} shows the model prediction
accuracy: the model prediction accuracy is very high for the data. 
We first trained the model on the applications which we are going to use for our simulations and 
see how it performed which are Random-Permutation, Shift-256, Stencil3d, Stencil4d, Milc, Nekbone and later on we added Subcom3d-a2a, Kripke , Laghos, SW4lite, and AMG for training to see if the model is able to handle the new data, along with the existing data and still provide a accurate prediction. We do this to make sure that our model is capable of getting  updated with new traffic as an when it comes. 

\begin{comment}
\paragraph{Model Offline Prediction}				
We conducted a study on three workloads generated by randomly selecting 
applications from Stencil4d, Subcom3d-a2a, Kripke, Laghos, AMG,
and SW4lite for  
ranks 32, 64, 128, 256, and 512. The TraceR-CODES simulator was used to 
simulate the corresponding workloads, and the flow data was collected.
The network flows that we gathered 
for the workload were predicted using the model. 

To validate the model's ability to forecast elephants and mice for any 
applications, we tested the model on unknown data. The model was employed to 
forecast the network flows of the application while it was operated 
individually and as a workload, and in each case, the model accurately 
predicted the flows by more than 95\%. The data used for making the 
predictions are presented in the table below.

Overall, our study demonstrates the effectiveness of the proposed neural 
network model for accurately predicting network flows of elephant and mice for 
various applications, which can assist in improving the performance of HPC 
systems.
%
\end{comment}

\begin{table}[h]
  \centering
  \caption{Accuracy of predicting elephants and mice flows by the DNN model for averaged across 32, 64, 128, 256 and 512 ranks}
    \label{tab:model_acc}
    \begin{tabular}{lr} \toprule
\multirow{1}{*}{Applications} & Accuracy \\ \midrule
Random-permutation & 100.0\\
Shift-256 & 100.0\\
Stencil3d & 100.0\\
Stencil4d & 100.0\\
Milc & 100.0\\
Nekbone & 100.0\\
\midrule
AMG & 100.0\\
Kripke & 100.0\\
Laghos & 100.0\\
Subcom3d & 100.0\\
SW4lite & 98.2\\ \bottomrule
\end{tabular}
\end{table}


\subsection{Flow identification with user input}

For static communications in HPC applications like the ones in Stencil4d
shown in Figure~\ref{code.stencil}, the HPC application developer (or
compiler and communication library) has
the knowledge whether a communication is an
elephant flow or not and can mark each flow in an MPI point-to-point
communication as elephant flow or a mice flow or all flows in an
MPI collective communication as elephant flows.
With this approach, the SDN basically passes the flow identification
task to the applications (done by application developer, compiler, or
runtime library), which greatly simplifies the SDN operation.

\begin{comment}
\textcolor{red}{
  The code indicates that each node sends eight communication messages to its neighbors. Additionally, since the neighbors are determined only during the compile time and the set of neighbors do not change during the runtime, user have the ability to correctly predict the communication characteristic. If the user knows the mapping of the ranks to nodes in the HPC system during the running of the application and also the data sent in the MPI calls, they can determine the communication characteristic of the application in the actual system and can determine which flows are elephants and which are mice when a HPC application starts running on a system.}
\end{comment}

There are different static communications in HPC applications. Consider
MPI applications. The most complete information about a point-to-point
communication includes the source MPI rank, the destination rank, and the
message size. The communications in Stencil4d belong to this type. 
For such communications, users can give the hint when the
application is loaded (when MPI ranks are mapped to physical nodes).
For communications whose source and destination cannot be determined
statically, but the size can be decided, the hints may be given at runtime
when the communications are executed.
The example of Laghos shown in Figure~\ref{code.laghos} belongs to this type. 
Such information can also be used
for collective communications where a group of processes are involved in
the communication.
In the case when message size cannot be determined, the user may still use the
API to indicate that a flow is an likely
elephant flow with the knowledge of the applications. The SDN may use a
simpler flow classification than the ones dealing with the most general
unknown flows.
 
\begin{comment}
Here, the system detects the elephant flows and checks
if it is marked by the user, if both marks the flows as elephant then the flow is regarded as elephant, however if the user indicated the flow wrongly, and the system, detects the flows as elephant flows for three or more consecutive intervals then the priority is given to the system clarification. Using user information, the system and the user work together
to identify flows and the flow identification at the network system level is
greatly simplified. 
  \textcolor{red}{ Figure~\ref{fig:complete_info} shows the Speedup with respect to D-mod-k routing for static applications for 256, 512 and 1024 ranks when users have the complete knowledge of the application characteristic. For Random permutation, having such knowledge beforehand offers a speedup of around 2 for 256 ranks. For larger ranks it still achieves a speedup around 1.5. For Stencil4d the approach provides a speedup of more than 1.4}


\begin{figure}[h]
  \centering
  \includegraphics[width=\columnwidth]{./figs/scripts/plot_scripts/complete-info.eps}
  \caption{Speedup in latency for various techniques compared with D-mod-k for 
applications with static communication with user marked elephant flow only on fat-tree for 256, 512 and 1024 processes.}
  \label{fig:complete_info}
\end{figure}

\end{comment}


\section{Phase Identification}
\label{sec:phase_identification}

HPC applications exhibit phased behavior with alternating
computation and communication phases. Since communications in different
phases do not overlap, network resources such as communication channels
allocated to communications in one phase can be reused for communications
in another phase. Hence, identifying communication phases, which is unique
in the HPC environment, allows the SDN to manage resources more effectively.

\subsection{Phase Identification With User Hints}

Communication phases in an HPC application can be easily identified
inside a program: there are often a section of code corresponding
for the communications in the program. Figure~\ref{code.laghos.1}
shows the code snippet for Laghos with communication phase marked. 
Here, after initiating MPI\_Allreduce
communications, the node performs MPI\_Barrier to make sure all ranks 
have the data, before initiating the computation phase where
it calculates the density.
Now, the communication pattern may be different in the two
communication phases in the given code as the comm world is calculated
at runtime. Being able to reuse network resources in different phases will
significantly improve network resource utilization. 
I propose to have an API for HPC application to give
information about phases. With the API, the user can insert a marker before
and after each communication
phase in the program to inform the network the starting and
ending of a communication phase. With the assistance from the user,
the SDN network can detect communication phases in an HPC application
without minor overheads: once all processes for an application
enter a computation phase, the application is in the computation phase and
resources for communications in the previous phase can be released; as
soon as one process enters a communication phase, the application enters
the communication phase. 

\begin{figure}[H]
\begin{lstlisting}[breaklines, language=C++, frame=single, tabsize=4, basicstyle=\ttfamily]
LagrangianHydroOperator (...){
        ...
        ParMesh *pm = H1FESpace.GetParMesh();
        MPI_Allreduce(&loc_area, &glob_area, 1, MPI_DOUBLE, MPI_SUM, pm->GetComm
());
        ...
}
\end{lstlisting}
\caption{Laghos code snippet}
\label{code.laghos.1}
\end{figure}

\begin{comment}
Communications in different phases do not happen at the same time. As such, network
resources (such as communication channels) can be allocated
independently in different phases.
Hence, accurately identifying communication phases can improve
the SDN support for HPC applications. Traditionally SDN schemes such
as the Hedera's threshold-based scheme automatically detect communication
changes when the flow statistics are collected and processed. However,
the granularity of such detection is the same as that for flow statistics
collection and processing, and network reconfiguration,
which is in the order in seconds. Such a coarse granularity is ineffective
for tightly coupled HPC applications whose phases can change in
subseconds or even submilliseconds. 
\end{comment}

%In this section I will talk about the phases of 
%HPC application and how to detect them during application runtime.

\subsection{Dynamic Communication Phase Identification}

\begin{comment}
 Flow information used to
classify elephant flows in current phase should ignore the flow information for the flows in the
previous phase as these two set of flows  don't occur together.
In order to
determine if the network is in a communication phase, the network must be
probed at intervals smaller than the polling interval to check if data is being
sent above a certain threshold. If it is, then the network is in a
communication phase; otherwise, it is mostly in a computation phase.
If the
network transitions from a computation phase to a communication phase or vice versa during
probing, previous flow paths for elephant flows are disregarded, and the flow
informations for each flow in the current phase is considered for
prediction. 
\end{comment}

Without the hints from application, communication phases in an HPC
application can be detected by finding the computation period in the
application when very few communications are performed. 
My dynamic communication phase identification algorithm
is shown in Figure~\ref{alg:phase_detection}. I first decide the interval
value for a minimal computation phase (e.g. 100ms). The phase detection
algorithm is executed periodically at the interval boundary to determine
whether phase is a communication phase or a computation phase depending
whether the total data communicated in the application during the phase
passes a threshold value. If a current phase is a computation phase and
the previous phase is a communication phase, network resources allocated
for the application in previous phases are released. 


\begin{algorithm}
\DontPrintSemicolon

\caption{Dynamic phase identification algorithm}
\label{alg:phase_detection}

Let Total\_data be all data sent by the application in the current interval\;
    \If{($Total\_data > Threshold$)}{
        Current phase is a communication phase\;
    }
    \Else{
      Current phase is a computation phase\;
      \If {(The previous phase is a communication phase)} {
        Release resources allocated in previous phases
        }
    }

\end{algorithm}



\section{SDN Routing}
\label{sec:sdn_routing}

In this work, I develop SDN-based routing techniques for both full-bisection fat-trees and tapered fat-trees, with support for single-path as well as multipath routing. My primary objective is to minimize congestion and balance link utilization—particularly for large elephant flows—in order to improve overall communication performance in high-performance computing systems.

Elephant flows are long-lived and bandwidth-intensive. When multiple such flows share the same network link, they contend for limited bandwidth and introduce significant congestion. This behavior makes elephant flows the primary contributors to network bottlenecks in HPC and datacenter environments. Accordingly, I define congestion as a measure of contention on a link, quantified by the number of active elephant flows traversing it.

Let the network be modeled as a directed graph \( G = (V, E) \), where \( V \) is the set of switches and terminals, and \( E \) is the set of directed links. Let \( F = \{f_1, f_2, \dots, f_n\} \) denote the set of elephant flows in the network. Each flow \( f \in F \) is assigned a single path \( P_f \subseteq E \) by the SDN controller.

For single-path routing, the congestion \( c(e) \) on a link \( e \in E \) is defined as:

\[
c(e) = \sum_{f \in F} \mathbf{1}_{\{e \in P_f\}}
\]

Here, \( \mathbf{1}_{\{e \in P_f\}} \) is an indicator function that equals 1 if link \( e \) is part of the path assigned to flow \( f \), and 0 otherwise.

The maximum link congestion across the network is then given by:

\[
C_{\text{max}} = \max_{e \in E} c(e)
\]

This congestion metric forms the basis for evaluating routing efficiency in my single-path SDN scheme. It guides the routing decision process by identifying and minimizing the worst-case contention experienced on any single link in the network.


\subsection{SDN-Singlepath Routing}

In single-path SDN routing, I assign each elephant flow a unique route through the network. Unlike traditional static routing methods, which rely on preconfigured tables, SDN enables dynamic path selection using a global view of current link utilization. This centralized perspective allows the controller to make informed decisions that minimize congestion and avoid oversubscription on critical links.

To explore the design space of single-path routing under SDN, I implement two complementary strategies: \textit{SDN-greedy} and \textit{SDN-optimal} which we talk about in the following sections.

\subsubsection{SDN-greedy}

\textit{SDN-greedy} is a lightweight, single-path routing algorithm designed to operate on both full-bisection and tapered fat-tree topologies. The algorithm leverages the SDN controller’s global view of the network to make path selection decisions that minimize link congestion in real time.

As shown in algorithm~\ref{alg:sdn_greedy}, the controller processes each elephant flow individually. For a given flow, it enumerates all feasible paths between the source and destination switches. For each candidate path, it computes the maximum link congestion \( c(e) \) across all links \( e \) on the path, which corresponds to the path's worst-case contention. This value is equivalent to evaluating the local \( C_{\text{max}} \) for that path. The controller then selects the path with the smallest such value and assigns it to the flow.

By repeating this process for all elephant flows in the current scheduling phase, the algorithm constructs a routing table that seeks to keep the network-wide maximum link congestion low. This greedy strategy effectively spreads traffic across the network to reduce the risk of bottlenecks. Due to its simplicity and responsiveness, SDN-greedy is well-suited for environments with light to moderate traffic.

However, because flows are routed independently, SDN-greedy may cause multiple flows to converge on the same links, especially under heavy or bursty traffic. This lack of global coordination can lead to suboptimal load balancing in dense communication scenarios.

\begin{algorithm}[H]
\DontPrintSemicolon
\caption{SDN-greedy algorithm}
\label{alg:sdn_greedy}
\KwInput{Elephant flows in a phase $EF$}
\KwOutput{Load-balanced routing table $Routing\_table$}
\SetKwFunction{FMain}{ComputeLoadBalancedRoutingTable}
\SetKwProg{Fn}{Function}{:}{}
\Fn{\FMain{$EF$}}{
    Initialize an empty map $Routing\_table$\;
    \ForEach{$E \in EF$}{
        Get current link loads in the network\;
        Get all possible paths for $Source$ and $Destination$ of $E$\;
        \ForEach{$path$ in $All\_possible\_paths$}{
            Compute maximum link congestion $c(e)$ for links in $path$\;
        }
        Add path with minimum $C_{\text{max}}$ to $Routing\_table$ for $E$\;
    }
    \KwRet{$Routing\_table$}\;
}
\end{algorithm}



\subsection{SDN-Optimal Routing Algorithm}

In fat-tree topologies, particularly full-bisection fat-trees, minimizing link contention is crucial to improving communication performance. In this work, I introduce a single-path routing algorithm, called \textit{SDN-optimal}, that achieves the theoretically lowest possible value of maximum link congestion, denoted by \( C_{\text{max}} \), across all network links. Specifically, SDN-optimal guarantees that \( C_{\text{max}} = 1 \) for any permutation of elephant flows in a full-bisection fat-tree. This is achieved by (1) partitioning the flows into a minimum number of disjoint permutations, and (2) scheduling each permutation such that no two flows share any link. The second step leverages the structural properties of full-bisection fat-trees and Hall’s Marriage Theorem to ensure contention-free routing.

\subsubsection{SDN-optimal Algorithm for Full-Bisection Fat-Trees}

The SDN-optimal algorithm consists of two main steps: (1) partitioning the traffic pattern into \( k \) disjoint permutations, and (2) routing each permutation using a contention-free schedule.

\paragraph{Step 1: Flow Partitioning into Permutations.}  
Given a set of elephant flows \( F \), I construct a bipartite graph \( G = (S, D, E) \), where \( S \) and \( D \) are source and destination nodes, and each edge \( e \in E \) represents a flow. To ensure that each source and destination has the same number of flows (i.e., a regular graph), I add dummy nodes and edges to make the graph \( k \)-regular, where \( k \) is the maximum degree in the original graph. I then apply edge-coloring (via Kőnig’s Theorem) to decompose the graph into \( k \) disjoint perfect matchings. Each matching corresponds to a permutation \( P_1, P_2, ..., P_k \), where each node appears exactly once per permutation. The algorithm is described in algorithm ~\ref{alg:partition_permutations}

\begin{algorithm}[H]
\DontPrintSemicolon
\caption{Flow Partitioning into Disjoint Permutations}
\label{alg:partition_permutations}

\KwInput{Flow set $F$, max degree $k$}
\KwOutput{$k$ disjoint permutations $P_1, ..., P_k$}

Construct bipartite graph $G = (S, D, E)$ from flows in $F$\;

Pad $G$ with dummy nodes/edges to make it $k$-regular\;

Apply edge-coloring to obtain $k$ disjoint matchings $P_1, ..., P_k$\;

Remove dummy flows from each $P_i$\;

\KwRet{$P_1, ..., P_k$}
\end{algorithm}

\paragraph{Step 2: Contention-Free Scheduling.}  
For each permutation \( P_i \), the flows are routed through the fat-tree using link-disjoint paths. In a full-bisection bandwidth of the topology, because the number of flows per permutation equals the number of available links at each network layer, Hall’s Marriage Theorem guarantees the existence of a perfect matching, no link will be used by more than one flow in a given permutation. Therefore, the maximum congestion per link in each permutation is exactly 1. The algorithm is described in algorithm ~\ref{alg:contention_free_scheduling}


\begin{algorithm}[H]
\DontPrintSemicolon
\caption{Contention-Free Scheduling for Permutation $P_i$}
\label{alg:contention_free_scheduling}

\KwInput{Permutation $P_i$, topology $G$, link usage table $L$}
\KwOutput{Routing paths for all flows in $P_i$}

\tcp{Step 1: Schedule Intra-Pod Flows}
\ForEach{intra-pod flow $f \in P_i$}{
  \ForEach{uplink $u$ from source leaf to aggregation}{
    \If{$u$ is unused}{
      Find corresponding downlink $d$ to destination leaf\;
      \If{$d$ is unused}{
        Assign path: leaf $\rightarrow$ agg $\rightarrow$ leaf\;
        Mark $u$, $d$ as used; \textbf{break}\;
      }
    }
  }
  \If{no path assigned}{
    Apply non-blocking rearrangement to free a valid intra-pod path\;
  }
}

\tcp{Step 2: Schedule Inter-Pod Flows}
\ForEach{inter-pod flow $f \in P_i$}{
  \ForEach{uplink $u_1$ from source leaf to aggregation}{
    \If{$u_1$ is unused}{
      \ForEach{uplink $u_2$ from aggregation to core}{
        \If{$u_2$ is unused}{
          Find downlink $d_2$ from core to destination agg\;
          \If{$d_2$ is unused}{
            Find downlink $d_1$ to destination leaf\;
            \If{$d_1$ is unused}{
              Assign full path: leaf $\rightarrow$ agg $\rightarrow$ core $\rightarrow$ agg $\rightarrow$ leaf\;
              Mark all links as used; \textbf{break}\;
            }
          }
        }
      }
    }
  }
  \If{no path assigned}{
    Apply non-blocking rearrangement to free a valid inter-pod path\;
  }
}
\end{algorithm}


\paragraph{Proof Sketch of Optimality.}

In a full-bisection fat-tree, the number of aggregation switches is equal to the number of leaf switches. Each leaf switch has \( \frac{K}{2} \) uplinks to aggregation switches. The total number of links between the leaf and aggregation switches is \( N_{\text{leaf}} \times \frac{K}{2} \). Since each compute node connects to a downlink port on a leaf switch, the total number of compute nodes is also \( N_{\text{leaf}} \times \frac{K}{2} \). In a full permutation traffic pattern, each compute node sends and receives exactly one flow, so the total number of flows in each permutation is equal to the number of compute nodes. Therefore, the number of flows in each permutation is equal to the number of links between the leaf and aggregation switches.

Similarly, the number of links between the aggregation and core switches is also equal to the number of flows in each permutation, because each aggregation switch has \( \frac{K}{2} \) uplinks to core switches. As a result, the number of flows in each permutation exactly matches the number of available links at each layer of the network.

This one-to-one correspondence between flows and available links allows the use of Hall’s Marriage Theorem to guarantee a contention-free assignment. Suppose, for contradiction, that in some permutation, a link is used by more than one flow. Then at least one other link must remain unused, violating the condition required for a perfect matching. This contradiction implies that no two flows in the same permutation can share a link. Therefore, the maximum congestion on any link in any permutation is 1. Since this holds for every permutation, the maximum link congestion across all permutations is also 1. Thus, the SDN-optimal algorithm guarantees \( C_{\text{max}} = 1 \) and achieves contention-free routing in a full-bisection fat-tree.

\subsubsection{SDN-Optimal for Tapered Fat-Trees}

In a tapered fat-tree, bandwidth reduction occurs at higher layers, such as the aggregate to core level has less links than leaf to aggregate level. This architectural tapering leads to insufficient link capacity when routing permutation traffic, where the number of flows often exceeds the number of available links at higher levels. As a result, flows begin to contend for shared links, and the assumptions of perfect matching guaranteed by Hall’s Marriage Theorem no longer hold. Consequently, achieving a contention-free assignment for all flows within a single permutation becomes infeasible.

To overcome this limitation and preserve optimal scheduling, the SDN-optimal routing strategy partitions the full permutation into multiple sub-permutations. Each sub-permutation is constructed such that the number of flows passing through each layer of the network matches the number of available links at that layer. This ensures that within each sub-permutation, contention-free routing is still possible using the techniques previously employed.

Our goal is to make sure that we create sub-permuatations by selects flows in such a way that the links in between each fat-tree layers have no contention and has the maximum utilization. 
To construct these sub-permutations, the algorithm first selects flows that traverse the core layer—typically inter-pod flows that consume both leaf-to-aggregate and aggregate-to-core links. It continues selecting such flows until all available core-level links are utilized. At this point, adding more inter-pod flows would introduce contention. The algorithm then fills the remaining capacity at the aggregation layer by selecting intra-pod flows, which consume only leaf-to-aggregate and aggregate-to-leaf links. The result is a sub-permutation that fully utilizes available link resources without exceeding capacity at any layer. This allows Hall’s Marriage Theorem to be applied to guarantee a conflict-free routing for each sub-permutation.

\begin{algorithm}[H]
\caption{Sub-Permutation Construction Using Core and Aggregate Link Counters}
\label{alg:sub_permutation}
\KwIn{
    $F_{\text{core}}$: set of flows that traverse the core switch\\
    $F_{\text{agg}}$: set of flows that only traverse the aggregate switch\\
    $c$: number of available core-to-aggregate links per sub-permutation\\
    $a$: number of available aggregate-to-leaf links per sub-permutation
}
\KwOut{
    $P_{\text{list}}$: list of sub-permutations
}

$P_{\text{list}} \gets \emptyset$\;

\While{$F_{\text{core}} \neq \emptyset$ \textbf{or} $F_{\text{agg}} \neq \emptyset$}{
    $P \gets \emptyset$\;
    $core\_links \gets c$\;
    $agg\_links \gets a$\;

    \tcp{Stage 1: Add core-level flows}
    \ForEach{$f \in F_{\text{core}}$}{
        Add $f$ to $P$\;
        $core\_links \gets core\_links - 1$\;
        Remove $f$ from $F_{\text{core}}$\;
        \If{$core\_links == 0$ \textbf{or} $F_{\text{core}} == \emptyset$}{
            \textbf{break}
        }
    }

    \tcp{Stage 2: Add aggregate-level flows}
    \ForEach{$f \in F_{\text{agg}}$}{
        Add $f$ to $P$\;
        $agg\_links \gets agg\_links - 1$\;
        Remove $f$ from $F_{\text{agg}}$\;
        \If{$agg\_links == 0$ \textbf{or} $F_{\text{agg}} == \emptyset$}{
            \textbf{break}
        }
    }

    Append $P$ to $P_{\text{list}}$\;
}

\Return $P_{\text{list}}$\;
\end{algorithm}


\subsection{SDN-Multipath Routing}

Multipath routing enables traffic to be split across multiple paths, improving bandwidth utilization and often reducing congestion. However, in many scenarios where flow paths overlap significantly, for example in shift traffic, this can introduce additional contention. In contrast, single-path routing (such as SDN-optimal) can perform better when flows are carefully scheduled to avoid bottlenecks in those cases.

To intelligently decide between these two strategies, I propose \textit{SDN-adaptive} routing for the multipath routing.
\subsubsection{SDN-adaptive}
The key idea behind SDN-adaptive is to first gather runtime information about the application’s traffic characteristics. During an initial profiling phase, SDN-adaptive collects communication performance data by running one iteration using adaptive multipath routing. This information is then sent to the SDN controller, which uses it decide the routing.
The process begins by executing one iteration of the application using adaptive multipath routing. The SDN controller records the communication time from this iteration as a reference. Then, the full simulation is restarted using SDN-optimal routing. After completing the first iteration of this SDN-optimal run, the controller compares the current communication time with the previously recorded time from the adaptive run.

If SDN-optimal demonstrates better performance (i.e., lower communication time), it is retained for the rest of the simulation. If, however, SDN-optimal is slower than adaptive, this indicates that multipath routing is more effective for the application's communication pattern. In that case, the controller immediately switches to adaptive routing for the remainder of the simulation.

This design allows the routing policy to be tuned to the observed behavior of the application, ensuring better adaptability across diverse workloads. The algorithm is described in algorithm ~\ref{alg:sdn_adaptive_api}

\begin{algorithm}[H]
\DontPrintSemicolon
\caption{SDN-adaptive algorithm}
\label{alg:sdn_adaptive_api}

\KwInput{%
  Application $A$; \\
  Topology $G$; \\
  Routing schemes: SDN-optimal and Adaptive
}
\KwOutput{%
  Final routing strategy and flow configuration
}

Run one iteration of $A$ using Adaptive routing\;
Record $\mathrm{commTime}_{\text{Adaptive}}$\;

Start full simulation of $A$ using SDN-optimal routing\;
After first iteration, record $\mathrm{commTime}_{\text{SDN-optimal}}$\;

\If{$\mathrm{commTime}_{\text{SDN-optimal}} < \mathrm{commTime}_{\text{Adaptive}}$}{
  Continue simulation with SDN-optimal routing\;
}
\Else{
  Switch to Adaptive routing for remainder of simulation\;
}
\end{algorithm}


\section{Performance Evaluation}
The proposed techniques have been extensively studied using
the TraceR-CODES simulator~\cite{jain2016evaluating,mubarak2016enabling}.
TraceR-CODES is a software tool suite used for
performance analysis of parallel and distributed applications, and it
is specifically designed to simulate large-scale scientific applications
running on high-performance computing systems. In the following, I will first
discuss the experimental setup and the extensions that are added to
TraceR-CODES to support the evaluation. After that, the performance results
will be presented.

\subsection{Experimental Setup}

The experiments are performed on two distinct Fat-tree configurations:
a 1024-node full-bisection Fat-tree and a 1536-node tapered Fat-tree
with a 3:1 oversubscription ratio. In both configurations, each switch
is equipped with 32 ports, and each leaf switch connects to 16 compute
nodes. Importantly, all 32 ports of the core and aggregate switches are
fully utilized, connecting exclusively to other switches to preserve the
hierarchical structure of the topology.

The proposed SDN techniques for HPC environments including flow
identification, phase identification, and SDN routing are added to
TraceR-CODES. Flow identification schemes with and without user input
described in Section~\ref{sec:flow_identification}, including the
threshold-based scheme, the DNN-based scheme, user-input based scheme
are incorporated in TraceR-CODES. 
To support the machine learning based flow identification scheme,
a trained DNN model has been integrated
into TraceR-CODES using Google TensorFlow's C APIs. To ensure
compatibility with the TensorFlow libraries, I utilized
MVAPICH 2 as the MPI implementation and adopted C++14 from GNU
version 9.0.1 as the compiler standard. The DNN model is able to 
process real-time network statistics and  
dynamically predicts network flows during the simulation runtime.
The user-input based scheme is supported by marking each MPI call with
user hints. 
%This capability not only enhances prediction accuracy but also enables
%real-time data processing, making the model both robust and efficient.

%In addition to evaluate SDN-based routing, I also investigate the impact of hardware design characteristics on performance using the Fat-tree connection topology. The Fat-tree topology is a well-established interconnect architecture commonly employed in HPC systems and data centers.

The functionality to support phase identification with and without user
information is added to the simulator and the simulator can be configured
to use a particular phase identification scheme. For routing, 
three types of SDN routing mechanisms are added: 
{\em SDN-greedy} that allocates paths by minimizing congestion in real time using simple heuristics, 
{\em SDN-optimal} that computes globally least-congested paths by evaluating all possible routes, and {\em SDN-adaptive} that dynamically adjusts
routing decisions based on the current traffic patterns and flow types.
Each of these methods is described in detail in earlier. Below is the
table outlining the network configuration parameters used in the simulation.

\begin{table}[h]
\centering
\caption{Network parameters for simulation of SHS}
\label{tab:params}
\vspace{1em}
\begin{tabular}{ll}
\toprule
Parameter & Value \\
\midrule
Packet Size     & 8192 Bytes \\
Switch Radix    & 32 \\
Link Bandwidth  & 11.9 GB/s \\
Eager Limit     & 64000 Bytes \\
NIC Scheduler   & Round-robin \\
\bottomrule
\end{tabular}
\end{table}

To mitigate the impact of various types of delay on our data, specifically in 
the communication aspect of the experiment, we have configured the router 
delay, network interface controller (NIC) delay, software delay, and remote 
direct memory access (RDMA) delay to zero. This setup allows us to eliminate 
any extraneous delay factors and isolate the effects of the communication 
process on our data analysis.


\subsection{Application and Workloads}
Eight representative applications are used in the study:
\texttt{Random-Permutation}, \texttt{Shift}, \texttt{Stencil3d}, \texttt{Stencil4d}, \texttt{Milc}, and \texttt{Nekbone}, \texttt{Random-Permutation-Mixed},
and \texttt{Stencil-Mixed}. All of the applications except \texttt{Random-Permutation-Mixed} and \texttt{Stencil-Mixed} including 
their computation and communication characteristics have been described
in detail in Chapter 2.

\begin{itemize}
\item \textbf{Random-Permutation-Mixed}: This pattern alternates between two distinct random-permutation communication phases, separated by a computation phase. Each communication phase uses a different source-destination mapping, introducing dynamic changes in the communication structure.

\item \textbf{Stencil-Mixed}: This application combines a \texttt{Stencil3d} phase followed by a \texttt{Stencil4d} phase, with an intermediate computation phase. The transition between patterns allows evaluation of the routing system’s responsiveness to changes in spatial communication demands.
\end{itemize}

All of these applications are communication-intensive, but 
they exhibit diverse communication behaviors ranging from synthetic
near-neighbor patterns to production-level scientific codes
and serve as appropriate test cases to assess how well SDN techniques
manage different types of network traffic.

To ensure accurate analysis of communication transitions, I enforce
explicit synchronization barriers between phases. This guarantees
that each communication phase is fully completed before the next begins,
preventing overlap and allowing isolated assessment of routing behavior across phases.

%\section{Performance Study}
The result of various SDN techniques which are used in HPC is presented here

\subsection{Evaluation of flow classification techniques}

In this section, we compare the impact of different flow identification techniques on the SDN algorithm for various applications under full bisection fat-tree and 3-to-1 tapered fat-tree configurations. The evaluation includes performance metrics across application communication time for different applications. To evaluate the flow detection technique, we kept the routing fixed as SDN‑optimal and used communication‑computation phase detection, while varying the flow detection methods.

\subsubsection{Performance Under Full Fat-Tree}
\begin{figure}[h]
  \centering
  \includegraphics[width=\columnwidth]{./figs_4/full_fat_flow_detection.pdf}
  \caption{Comparison of flow detection in full bisection fat-tree of 1024 nodes}
  \label{fig:fld_full}
\end{figure}

The figure ~\ref{fig:fld_full} compares communication speedup achieved by three flow detection techniques, SDN User, SDN DNN, and SDN Threshold across various applications, including Random-Permutation, Random-mixed Shift, Stencil3D, Stencil4D, Stencil-mixed, Milc, and Nekbone. These applications represent diverse traffic patterns, ranging from simple to highly complex workloads, making them ideal benchmarks for evaluating network performance.  A higher bar (speedup) signifies better efficiency and lower communication time relative to the baseline. The user flow identification technique performs best due to early flow identification, enabling prompt traffic management. DNN detection ranks second, limited by a 0.3-millisecond delay. Threshold detection performs worst due to delayed flow recognition. Importantly, all three techniques perform better than the widely used single-path static routing, demonstrating the effectiveness of dynamic flow detection in improving communication efficiency within SDN frameworks. User detection consistently leads across all applications. DNN detection shows reasonable gains despite initial delay. Threshold detection offers minimal improvement. These results highlight the importance of incorporating flow detection mechanisms to enhance SDN routing performance beyond conventional static routing methods like D-mod-K, also, the DNN model does a faster flow classification compared to threshold based model without losing in performance.


\begin{comment}
\begin{figure*}[t]
  \centering
  \includegraphics[width=\textwidth]{./figs_4/combined_max_load_plot_full.pdf}
  \caption{Max data sent through a port in a switch level for all switches in full bisection fat-tree of 1024 nodes}
  \label{fig:ld_full}
\end{figure*}



The figure ~\ref{fig:ld_full} shows maximum outgoing data (Bytes) through ports across six application sections: Random-Permutation, Shift, Stencil3D, Stencil4D, MILC, and Nekbone. Each section contains bars representing three configurations:
Green represents the maximum load at the leaf switches, Blue indicates the maximum load at the aggregate switches, and Red signifies the maximum load at the core switches. Observations reveal that the Random-Permutation application shows the highest load at the leaf switch under the Static configuration. Similarly, Stencil3D and Stencil4D exhibit high loads at both leaf and aggregate switches in Static and Adaptive configurations. Nekbone records the highest outgoing data, particularly at the leaf switch under the Static configuration. When comparing SDN loads, they are consistently lower than those in Static configurations and remain lower than Adaptive configurations when communication density is moderate, as observed in applications like Random-Permutation, Shift, Stencil3D, and Stencil4D. SDN effectively distributes loads across ports, resulting in a positive speedup over static routing.
\end{comment}

\subsubsection{Performance Under Tapered Fat-Tree}
\begin{figure}[h]
  \centering
  \includegraphics[width=\columnwidth]{./figs_4/taper_fat_flow_detection.pdf}
  \caption{Comparison of flow detection in 3 to 1 taper fat-tree of 1536 nodes}
  \label{fig:fld_taper}
\end{figure}

The figure ~\ref{fig:fld_taper} illustrates the communication speedup achieved by different flow detection techniques (SDN User, SDN DNN, and SDN Threshold) under a 3-to-1 taper fat-tree topology with 1536 nodes and a 3:1 tapering ratio. This topology emphasizes the impact of reduced bandwidth at higher levels of the Fat-Tree structure.


In a tapered fat-tree topology, even a random-permutation pattern represents a relatively dense communication workload. To better demonstrate the effectiveness of our routing mechanism in balancing traffic, we introduced Random-Permutation-Third by removing two-thirds of the communication from a random-permutation of 1536 nodes, retaining only eight out of the 24 communication flows passing through a leaf router. Our evaluation revealed that SDN User consistently achieved the highest speedup, exceeding 6x, highlighting the benefits of early user-provided flow identification that enables optimal traffic balancing from the start. In contrast, SDN DNN and SDN Threshold exhibited moderate speedups of approximately 1.5x and 1.2x, respectively, due to delayed flow detection. Similar trends were observed in the Shift application, where SDN User maintained its superior performance, while SDN DNN and SDN Threshold provided moderate improvements over static routing. In compute-heavy applications such as Milc and Nekbone, the performance gap between techniques narrowed, though SDN User remained the top performer. The results suggest that compute-heavy traffic benefits from steady-state optimizations, though early flow detection consistently outperformed or matched the threshold-based approach. These findings underscore the critical role of early flow detection in achieving optimal traffic balancing, particularly in tapered Fat-Tree topologies with constrained bandwidth, as reflected by higher speedup values relative to baseline D-mod-K routing.


\begin{comment}
\begin{figure*}[t]
  \centering
  \includegraphics[width=\textwidth]{./figs_4/combined_max_load_plot_taper.pdf}
  \caption{Max data sent through a port in a switch level for all switches in 3-to-1 taper fat-tree of 1536 nodes}
  \label{fig:ld_taper}
\end{figure*}


In tapered, the aggregate switch ports have a lot of load, and in all cases, SDN with all various flow identification techniques have always drtibuted this load across the the other switch ports. SDN performance in load distribution is almost same as adaptive and in cases for Shift and Random-Permutation-Third it is doing a slightly better job than adaptive.



%
%
%
%
%
%          next Sub Section 2 of results
%
%
%
%
%

\end{comment}

\subsection{Evaluation of Phase Identification}
This section evaluates the effectiveness of phase identification, across varioous applications under full bisection fat-tree and 3-to-1 tapered fat-tree configurations. To evaluate the communication‑computation phase identification technique, we kept the routing fixed as SDN‑optimal and used the SDN user flow detection method, while varying the phase identification methods.

\subsubsection{Performance Under Full Fat-Tree}


\begin{figure}[h]
  \centering
  \includegraphics[width=\columnwidth]{./figs_4/phase_full.pdf}
  \caption{Comparison of phase identification in full bisection fat-tree of 1024 nodes}
  \label{fig:phase_full}
\end{figure}
The figure~\ref{fig:phase_full} shows phase identification results for full Fat-tree, where the x-axis represents different workloads, including Random-permutation, Random-mixed, Shift, Stencil3d, Stencil4d, Stencil-mixed, Milc, and Nekbone, while the y-axis quantifies the relative speedup. 
In Full Fat-Tree routing, phase-based optimization leverages threshold-based flow classification and SDN routing to improve communication efficiency. The analysis of communication speedup reveals notable improvements in random-mixed, stencil-mixed, and nekbone, with random-mixed achieving a 7.98\% increase in performance. The phase identification mechanism efficiently distinguishes between different communication phases. When a phase transition occurs, it promptly loads the corresponding routing table, ensuring seamless adaptation to dynamic communication patterns.

The evaluation time for each phase is one-tenth of a polling phase, during which the phase identifier continuously monitors network flows to detect transitions between computation and communication phases. Upon detecting a phase change, it swiftly loads the appropriate routing table to optimize data flow. In random-mixed, stencil-mixed, and nekbone, frequent transitions between computation and communication phases trigger rapid routing table updates, leading to measurable performance improvements in these applications.


\subsubsection{Performance Under Tapered Fat-Tree}
\begin{figure}[h]
  \centering
  \includegraphics[width=\columnwidth]{./figs_4/phase_taper.pdf}
  \caption{Comparison of phase identification in full bisection fat-tree of 1536 nodes}
  \label{fig:phase_taper}
\end{figure}
The figure~\ref{fig:phase_taper} shows phase identification results for full Fat-tree,
In Tapered Fat-Tree routing, phase-based optimization dynamically adjusts traffic injection rates using threshold-based flow classification and SDN routing to further enhance communication efficiency. The analysis of communication speedup shows notable improvements in random-mixed, stencil-mixed, and nekbone, with random-mixed achieving an 11.13\% increase in performance, surpassing the gains observed in Full Fat-Tree routing. The phase identification mechanism efficiently detects phase transitions and dynamically manages traffic flow to reduce network congestion

\subsection{Evaluation of SDN-based Routing}
This section evaluates the effectiveness of SDN-based routing approaches across various applications under full bisection fat-tree and 3-to-1 tapered fat-tree configurations of user identification of flows. We analyze performance in terms of communication times.To evaluate the various SDN routing techniques, we kept the communication‑computation phase identification fixed and used the SDN user flow detection method, while varying the SDN routing methods.

\subsubsection{Performance Under Full Fat-Tree}
The figure ~\ref{fig:routing_full} illustrates the communication speedup achieved by three routing techniques—Adaptive, SDN User, and SDN-Adaptive across various applications under a full fat-tree topology. The y-axis represents the speedup relative to static D-mod-K routing, where higher bars indicate better performance.


\begin{figure}[h]
  \centering
  \includegraphics[width=\columnwidth]{./figs_4/routing_full.pdf}
  \caption{Comparison of routing techniques in full fat-tree of 1024 nodes}
  \label{fig:routing_full}
\end{figure}
Single-path SDN routing consistently outperforms single-path DmodK routing across all scenarios.

While SDN routing generally performs better than Adaptive routing, the multipath nature of Adaptive routing allows it to balance the load more effectively when communication becomes dense. At this point, SDN-Adaptive, a hybrid approach combining SDN and Adaptive strategies, performs similarly to Adaptive routing at high communication density application and similar to SDN then the communication density is low, leveraging its flexibility to adapt based on the application's requirements.

\begin{comment}
\begin{figure}[t]
  \centering
  \includegraphics[width=\columnwidth]{./figs_4/two_column_multiplot_boxplots_full.pdf}
  \caption{Distribution of flow completion times in full fat-tree of 1024 nodes}
  \label{fig:flow_dist_full}
\end{figure}


The figure ~\ref{fig:flow_dist_full} show the distribution of distribution of flows based on flow completion times. 
\end{comment}

SDN even being a single path routing is kind of making sure that flows are completing together and there is no stragglers left behind. 


\subsubsection{Performance Under Tapered Fat-Tree}



\begin{figure}[h]
  \centering
  \includegraphics[width=\columnwidth]{./figs_4/routing_taper.pdf}
  \caption{Comparison of routing techniques in 3 to 1 taper fat-tree of 1536 nodes}
  \label{fig:routing_taper}
\end{figure}


The figure ~\ref{fig:routing_taper} presents a comparison of routing techniques (Adaptive, SDN User, and SDN-Adaptive Use) across three applications (Random-Permutation, Shift, and Milc) under a Tapered Fat-Tree topology. The y-axis represents the communication speedup relative to DmodK routing, where higher bars indicate better performance. In a Tapered Fat-Tree topology, applications typically experience dense traffic patterns due to limited bandwidth at higher levels caused by tapering. Adaptive routing demonstrates strong performance in balancing network load, especially under traffic-heavy scenarios. However, in Random-Permutation-Third, where traffic is reduced, our evaluation clearly shows that SDN-based routing outperforms adaptive routing. In Shift traffic patterns, where traffic predictability is higher, Adaptive routing effectively balances the load better than static routing, though SDN-based routing still makes slightly better decisions due to its global network awareness. SDN-Adaptive routing leverages the strengths of both Adaptive and SDN-based routing by dynamically adapting to application-specific traffic patterns, selecting the most suitable approach for optimal performance. 

\begin{comment}
\begin{figure}[t]
  \centering
  \includegraphics[width=\columnwidth]{./figs_4/two_column_multiplot_boxplots_taper.pdf}
  \caption{Distribution of flow completion times in 3-to-1 taper fat-tree of 1536 nodes}
  \label{fig:flow_dist_taper}
\end{figure}


The figure ~\ref{fig:flow_dist_full} illustrates the flow completion time distribution. Despite utilizing a single-path routing approach, SDN effectively ensures that flows complete within a similar timeframe, minimizing the presence of stragglers. This highlights the capability of SDN to maintain consistent flow completion, reducing delays and enhancing overall network performance.
\end{comment}

\section{Summary}
This chapter explores the adaptation of software-defined networking (SDN) techniques to the distinct communication patterns of high performance computing (HPC) applications. While SDN has been successful in traditional data center and wide-area network environments, its use in HPC systems has remained limited due to challenges posed by fine-grained and phase-driven communication behavior. This study addresses those limitations by introducing novel flow classification and communication phase identification mechanisms specifically designed to align with HPC workloads.

The core of this work involves a detailed simulation-based evaluation using the TraceR-CODES framework. Three flow classification approaches are considered: threshold-based classification, deep neural network (DNN)-based classification, and user-input-based classification. These are applied in conjunction with three SDN routing strategies: greedy routing, optimal routing, and adaptive routing. The techniques are evaluated across real and synthetic HPC workloads using two common network topologies: a 1024-node full-bisection fat-tree and a 1536-node 3-to-1 tapered fat-tree.

The results show that the strategies implemented in SDN routing perform very well under single-path routing. They significantly reduce network congestion and ensure the efficient execution of HPC applications within SDN-enabled systems. The multipath variant, SDN-adaptive, dynamically selects between SDN and adaptive routing and achieves performance that is either better than or comparable to adaptive routing alone, depending on the type of application and the density of communication. Furthermore, the use of phase-aware routing and early flow identification, particularly through user input or DNN-based models, plays a key role in adapting routing behavior to dynamic traffic patterns and improving overall communication efficiency.

By aligning SDN’s programmability and global control with the communication characteristics of HPC workloads, this work demonstrates that SDN can serve as a robust and efficient networking solution for modern supercomputing platforms. These contributions support the broader integration of SDN in HPC systems and provide a solid foundation for future research in intelligent, adaptive network control.

