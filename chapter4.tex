\chapter{Design and Evauation of Techniques for HPC platforms with SDN-capable Interconnects} In
this study, we tackle the challenge of incorporating Software Defined
Networking (SDN) for optimally running High-Performance Computing (HPC)
applications. As more HPC environments have SDN capabilities, there's a growing
need to manage the HPC applications communication across the underlying network
efficiently \cite{kreutz2014software, alalmaei2020sdn, he2016firebird,}. 
SDN helps by making the network smarter, so it can handle
the intense network resource demands of HPC applications better. 


In our work, we try to accomplish on the below objectives:
\begin{itemize}
\item We propose to have an network API to
allow users to indicate whether a communication is an elephant flow or not. This
would release the network system from classifying such flows.

\item We want to create a three routing algorithm which we can implement inside our
SDN controller, these routing functions help to load balance the network traffic
by efficiently scheduling the bandwidth heavy elephant flows which are classfied by our API.


\item Finally, we want to do a thorough simulation based evaluation of our techinques developed
in different node to rank mapping configurations.
\end{itemize}

In the following sections we propose about the details of flow identification, SDN-enabled flow scheduling,
topology and job mappings which we are going to do in our work. 


\section{Flow Identification}
The general
workload characteristics of an HPC application are vastly different from
traditional data center applications. First, HPC applications are often
massively parallelized, with various programming paradigms and libraries such as
Message Passing Interface (MPI) \cite{forum1994mpi}. The huge amuunt of computation that is
required in the application is divided into smaller computations and runs on
multiple computing nodes. Second, many HPC applications are tightly coupled –
there are dependencies among the parallel processes and the processes need to
communicate with each other during execution. Since many HPC applications
simulate physical phenomena in many iterations, the applications exhibit phased
behavior during execution with alternating communication and computation phases.
Communications in different iterations are often the same or have similar
characteristics. One common communication pattern for HPC applications is
nearest-neighbor where all the ranks which are spatially near each other take
part in communication. 

Our main objective is to harness the power of SDN to run HPC application which are 
very different from data center application in Data Centers.

SDN usually has two kinds of flows elephant flows that
carry large amounts of data from a source to a destination. These flows
constitute a bulk of the network traffic and are sensitive to network bandwidth.
Mice flows on the other hand carry small amount of data from source to
destination \cite{yang2020flow, afek2015sampling}, and are latency sensitive.

For HPC applications like the ones in Stencil4d shown in Figure \ref{code.stencil}, SDN
users (HPC application developers, compiler, or communication library) have the
knowledge whether a communication is an elephant flow or not. The code indicates
that each node sends eight communication messages to its neighbors.
Additionally, since the neighbors are determined only during the compile time
and the set of neighbours do not change during the runtime, the user has the
ability to correctly predict the communication characteristic. If the user knows
the mapping of the ranks to nodes in the HPC system during the running of the
application and also the data sent in the MPI calls, they can determine the
communication characteristics of the application in the actual system, and can
determine which flows are elephants and which are mice when a HPC application
starts running on a system. 
Even for some communications where the
destination is unknown such as the Lagos shown in
Figure \ref{code.laghos}. The API can be used to indicate that the unknown destination flow is
an elephant flow as long as the user can determine that the message size of the
communication is sufficiently large.
To identify an elephant flow, the information needed is the message size of the
flow. Therefore, we run an application initially with a certain node to rank mapping, and collect the
amount of data sent in each flow. We store this data in a matrix and then use a program to filter
flows which send more than a certain threshold of data across the entire simulation. 
These flows are classified as elephant flows. Our API, gurantees that the user has ground truth about
the communication which happens in a network when a specific set of HPC applications are run on that 
system.

\begin{figure}[hbtp]
\caption{Stencil4d Code snippet}
\label{code.stencil}
\begin{lstlisting}[breaklines]
for (int i = 0; i < MAX_ITER; i++) {
        MPI_Isend(sendn, 100000000, MPI_CHAR, north, 9, MPI_COMM_WORLD, &reqs[0]
);
        ...
        MPI_Irecv(recvn, 100000000, MPI_CHAR, north, 9, MPI_COMM_WORLD, &reqs[8]
);
        ...
        MPI_Waitall(16, req status);
}
\end{lstlisting}
\end{figure}



\begin{figure}[hbtp]
\caption{Laghos Code snippet}
\label{code.laghos}
\begin{lstlisting}[breaklines]
LagrangianHydroOperator (...){
        ...
        ParMesh *pm = H1FESpace.GetParMesh();
        MPI_Allreduce(&loc_area, &glob_area, 1, MPI_DOUBLE, MPI_SUM, pm->GetComm
());
        ...
}
\end{lstlisting}
\end{figure}


\section{SDN-enabled flow scheduling}
We propose to develop routing algorithm which aims to load balance the network traffic using the 
global view of a network through the SDN controller.
We intially use a greedy approach which calculates the maximum link load for all available
paths for each elephant flow, and then adds whichever path has the minimum of the maximum link load 
-that is adds the path which is least heavily loaded for routing the elephant flow. 
The subnet manager provided by the SDN accumulates information
about the network state and flows over the polling interval. If the SDN
scheduler has no prior knowledge about the network flows, it uses D-mod-k routing.
At each polling interval, the traffic classification provided by the user is forwarded to the SDN scheduler 
to obtain a load-balanced routing table for the subsequent interval. This load balance schedule is a single 
path schedule. We also propose to develop a multipath routing, where we analyse the buffer load for k least loaded
path, and use the one which has the lowest buffer load among the k paths to route our network flow. Finally,
we try to develop an optimal single path routing for our research, which tries to find a non-blocking path for 
every elephant flow in a permutation in a full bisection bandwidth fat-tree.


\section{Topology}
We evaluate our three SDN enhanced routing schemes, two single path routing and one multipath
routing, with both the widely popular static routing scheme in fat-tree called D-mod-k and the adaptive 
routing which is used in fat-trees. All our routing is schemes are evaluted on two different size
topologies, one mid-sized topology of 512 compute nodes and large-sized topology of 1024 compute
nodes. We evalute the developed techniques across 5 different applications, which are random permutation,
near neighbour, nekbone, lammps and milc; we discussed the applications in chapter 2. 

\section{Job Mapping}
We explore the influence of node-to-rank mapping in our HPC system applications, with particular emphasis on two methods: linear mapping and random mapping. Linear mapping organizes job ranks on nodes sequentially, maintaining spatial proximity between ranks. However, due to node fragmentation in real systems, complete linear mapping isn't always feasible, leading to a blend of linear and random mapping for job ranks. Consequently, it becomes necessary to assess the effects of random mapping, where job ranks are assigned to available nodes randomly. This random assignment increases spatial distance between job ranks, consequently intensifying network resource usage for communication. To facilitate this evaluation, we interface our preferred mapping with the simulator via an API compatible with the simulator's network model

