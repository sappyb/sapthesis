The SDN controller has the information of the traffic in
the network. It then uses the traffic information and network information
to schedule the traffic. I call the scheduling algorithm {\em SDN-based
routing}. Clearly, SDN-based routing can be different
for different network topologies and is constrained by the underlying
network routing mechanism such as single-path routing or adaptive routing. 
In this work, I develop SDN-based routing for both full-bisection bandwidth
fat-trees and tapered fat-trees, with support for single-path as well as
adaptive routing.

I consider routing (scheduling) elephant flows that are long-lived
and bandwidth-intensive. In the discussion, I will
model the system as a directed
graph \( G = (V=S\cup N, E) \), where $N$ is the set of compute nodes,
$S$ is the set of switches, and $V$ is the set of switches ($S$) and
compute nodes ($N$), and \( E \) is the set of directed links.
The traffic demand is represented by a set of
elephant flows with each flow being denoted as a source-destination pair
$f_i = (s_i, d_i)$, where $s_i, d_i \in N$.
\[D = \{f_1=(s_1, d_1), f_2=(s_2, d_2), ..., f_n = (s_n, d_n)\}\] 

Given a traffic demand $D$, the objective of an SDN-based routing scheme is to
achieve load balancing. In the following, I will first present my SDN-based
routing for networks with single path routing and then discuss the
scheme for networks with adaptive routing.

\subsection{Single path routing}

For single-path routing, all packets of a flow
follow the same path. For a traffic demand
$D = \{f_1=(s_1, d_1), f_2=(s_2, d_2), ..., f_n = (s_n, d_n)\}$, 
I will denote the path for flow $f_i$ to be $p_i$, which consists of a set
of directed links. For a given routing $R$,
let $L_l$ be the set of flows that are routed through link $l$.
The load of the link $l$ is the number of flows using the link ($|S_l|$).
The load of a path $p$ is the maximum load of the loads on all links
in the path: $L_p = \max_{l \in p} {|S_l|}$. $L_p$ is referred to as the path
load. The load of the network for traffic demand $D$
is the maximum of loads among all links:
\[L_D = \max_{l\in E} {|S_l|}\]

The design objective of my SDN-based single-path routing schemes is to
minimize $L_D$ for traffic demand $D$. My first algorithm,
\textit{SDN-greedy}, is a heuristic algorithm while the second algorithm,
\textit{SDN-optimal}, is optimal for full-bisection fat trees. 

\subsubsection{SDN-greedy}

\textit{SDN-greedy} is a lightweight, single-path routing algorithm
designed to operate on both full-bisection and tapered fat-tree
topologies. As shown in Algorithm~\ref{alg:sdn_greedy}, the controller
processes each elephant flow in order. For each flow, it enumerates
all feasible paths between the source and the destination. For each
candidate path, it computes the maximum link load for the path.
The controller then selects the path with the smallest maximum link load
and assigns path to the flow.
By repeating this process for all elephant flows in the current
scheduling phase, the algorithm constructs a routing table that seeks
to keep the network-wide maximum link congestion low. This greedy
strategy effectively spreads traffic across the network. However, this
greedy algorithm may not result in the optimal scheduling for a
traffic demand, especially when the traffic is dense. 

\begin{algorithm}[H]
\DontPrintSemicolon
\caption{SDN-greedy routing}
\label{alg:sdn_greedy}
\KwInput{Elephant flows in a phase $D$}
\KwOutput{Routing table $Routing\_table$ for all flows in $D$}
\SetKwFunction{FMain}{SDN-greedy}
\SetKwProg{Fn}{Function}{:}{}
\Fn{\FMain{$D$}}{
    Initialize an empty map $Routing\_table$\;
    \ForEach{$f_i = (s_i, d_i) \in D$}{
        Get current link loads in the network\;
        Compute all possible paths for $f_i$\;
        Compute the path load for each of the possible paths based
        on the current link loads of the network\;
        Add the path with minimum path for $f_i$ to
        $Routing\_table$ for $f_i$\;
    }
    \KwRet{$Routing\_table$}\;
}
\end{algorithm}

\subsection{SDN-optimal}

While \textit{SDN-greedy} spreads the flows to paths with minimum path
loads in a greedy manner, and does not guarantee to achieve optimal $L_D$ for
a given set of elephant flows, \textit{SDN-optimal} considers the set of flows
as a whole and achieves optimal $L_D$ for full-bisection fat trees.
In the following, I will first describe SDN-optimal for full-bisection fat
trees. After that I will discuss how to extend it for tapered fat trees. 

Given traffic demand
$D = \{f_1=(s_1, d_1), f_2=(s_2, d_2), ..., f_n = (s_n, d_n)\}$, I define
$SRC_{s\in N} = \{f_i = (s_i, d_i) | s_i = s\}$ and $DST_{d\in N} =
\{f_i = (s_i, d_i) | d_i = s\}$.  $SRC_s$ is the set of flows in $D$ whose
source node is $s$ and $DST_d$ is the set of flows in $D$ whose destination
node is $d$. I define the node load for a given $D$ as 
\[NL_D = \max(\max_{s \in N} \{|SRC_s|\}, \max_{d \in N} \{|DST_d|\})\]

The node load for a traffic demand is either the maximum number of flows
coming from the same source or the maximum number of flows going to
the same destination in the traffic demand. In a fat tree topology, since
each compute node connects to one link to a switch, I have the following
theory.

{\bf Theorem 1}: For a fat-tree topology and a given traffic demand $D$,
for any single path routing scheme, $L_D \ge NL_D$.

{\em Proof:} Since each compute
node only connects one out-going link and one incoming link. Any single path
routing scheme must route the flows from a source to the out-going link that
connects to it and route the flows to a destination to the in-coming link
to the destination. Hence,
\[L_D = \max_{l\in E} {|S_l|} \ge \max_{l\ is\ a\ out-going\ link\ of\ n\in N} {|S_l|}\] and 
\[L_D = \max_{l\in E} {|S_l|} \ge \max_{l\ is\ an\ incoming\ link\ of\ n\in N} {|S_l|}.\]
Hence, 
\[L_D = \max_{l\in E} {|S_l|} \ge \max(\max_{s \in N} \{|SRC_s|\}, \max_{d \in N} \{|DST_d|\}) = NL_D.\] $\Box$

\textit{SDN-optimal} consists of two steps. Given a traffic demand $D$,
the first step is to partition $D$ into $NL_D$ permutations. In the second
step, the algorithm uses a contention free scheduling algorithm to schedule
each of the $NL_D$ permutations. Since with a contention free scheduling
algorithm, each permutation can be routed with no link contention in a
full-bisection tree. Hence, for each permutation, the maximum link load among
all links in the network
is at most 1. As a result, scheduling $NL_D$ permutations will
have a maximum link load among all links in the network of $NL_D$. In other
words, with \textit{SDN-optimal}, $L_D \le NL_D$. Hence, \textit{SDN-optimal}
is optimal for any traffic demand on a full bisection fat tree. The following
theorem summarizes this discussion. 

{\bf Theorem 2}: For a full bisection fat tree and
\textit{SDN-optimal}, for any traffic demand $D$, $L_D = NL_D$:
\textit{SDN-optimal} is optimal for the full bisection fat tree. 
$\Box$

Next, I
will give details of the two steps in the algorithm.

\paragraph{Step 1: Partitioning $D$ into $NL_D$ permutations.}
Given the traffic demand $D$, the algorithm first constructs a
bipartite graph \( G = (S, D, E) \), where \( S \) and \( D \) are
source and destination nodes in $D$, and each edge \( e \in E \) represents
a flow: if $(s, d) \in D$, then there is an edge from $s\in S$ to $d\in D$. 
This initial bipartite graph is then augmented to be a $NL_D$-regular
multi-bipartite graph by adding dummy nodes and edges. Multi-bipartite graph
allows multiple edges between two nodes. I first add dummy nodes such that
$|S|$ is the same as $|D|$. After that, I add dummy edges to make each node
in $S$ has a degree of $NL_D$ and each node in $D$ has a degree of $NL_D$.
For each node in $|S|$, if its degree is less than $NL_D$, I find a node
in $D$ whose degree is less than $NL_D$ and add a dummy edge between the
two node. This process is repeated until all nodes have $NL_D$ degree. Once
the $NL_D$-regular multi-bipartite graph is build, 
the algorithm then applies edge-coloring (via Kőnig’s Theorem)
to decompose the graph into \( NL_D \) disjoint perfect matchings ~\cite{konig1916graphen}. Each matching corresponds to a permutation
\( P_1, P_2, ..., P_{NL_D} \). The dummy edge is then removed from the
obtained permutation to yield the final permutations.
The algorithm is described in Algorithm ~\ref{alg:partition_permutations}

\begin{algorithm}[H]
\DontPrintSemicolon
\caption{Flow partitioning into disjoint permutations}
\label{alg:partition_permutations}

\KwInput{Flow set $D$}
\KwOutput{$NL_D$ disjoint permutations $P_1, ..., P_{NL_D}$}

Construct bipartite graph $G = (S, D, E)$ from flows in $F$\;

Pad $G$ with dummy nodes and edges to make it $NL_D$-regular multi-bipartite
graph\;

Apply edge-coloring to obtain $NL_D$ disjoint matchings $P_1, ..., P_{NL_D}$\;

Remove dummy flows from each $P_i$\;

\KwRet{$P_1, ..., P_k$}
\end{algorithm}

\paragraph{Step 2: Contention-free scheduling for each permutation}  

Finding a contention-free schedule for a permutation on a full-bisection
fat-tree topology has been investigated for more than two decades, and a
number of efficient algorithms are now available\,%
\cite{Paull1962,Rodriguez2009,Zahavi2010,Prisacari2013}.  In this work, I
represent the permutation with a connection
matrix,\cite{Paull1962} and apply the coloured-matrix algorithm of
Rodríguez \emph{et al.}\,\cite{Rodriguez2009}, thereby ensuring that each
permutation is routed without link contention.

\subsubsection{SDN-optimal for Tapered Fat-Trees}

In a tapered fat-tree, bandwidth reduction occurs at higher layers,
such as the aggregate to core level has less links than leaf to aggregate
level. This architectural tapering leads to insufficient link capacity
when routing permutation traffic, where the number of flows often
exceeds the number of available links at higher levels.
As a result, \textit{SDN-optimal} does not work on tapered fat-trees
since achieving a contention-free assignment for all flows within a
single permutation becomes infeasible.

To overcome this limitation, the SDN-optimal routing strategy
partitions the full permutation into multiple sub-permutations.
Each sub-permutation is constructed such that the number of flows
passing through each layer of the network matches the number of
available links at that layer. This ensures that within each
sub-permutation, contention-free routing is still possible using
the techniques previously employed.

My goal is to make sure that I create sub-permuatations by
selecting flows in such a way that the links in between each
fat-tree layers have no contention and has the maximum utilization. 
To construct these sub-permutations, the algorithm first selects
flows that traverse the core layer, typically inter-pod flows that consume both leaf-to-aggregate and aggregate-to-core links. It continues selecting such flows until all available core-level links are utilized. At this point, adding more inter-pod flows would introduce contention. The algorithm then fills the remaining capacity at the aggregation layer by selecting intra-pod flows, which consume only leaf-to-aggregate and aggregate-to-leaf links. The result is a sub-permutation that fully utilizes available link resources without exceeding capacity at any layer. This allows Hall’s Marriage Theorem \cite{cameron2025hall} \cite{hall1987representatives} to be applied to guarantee a conflict-free routing for each sub-permutation.

\begin{algorithm}[H]
\caption{Sub-permutation construction}
\label{alg:sub_permutation}
\KwIn{
    $F_{\text{core}}$: Set of flows that traverse the core switch\\
    $F_{\text{agg}}$: Set of flows that only traverse the aggregate switch\\
    $c$: Number of available core-to-aggregate links per sub-permutation\\
    $a$: Number of available aggregate-to-leaf links per sub-permutation
}
\KwOut{
    $P_{\text{list}}$: list of sub-permutations
}

$P_{\text{list}} \gets \emptyset$\;

\While{$F_{\text{core}} \neq \emptyset$ \textbf{or} $F_{\text{agg}} \neq \emptyset$}{
    $P \gets \emptyset$\;
    $core\_links \gets c$\;
    $agg\_links \gets a$\;

    \tcp{Stage 1: Add core-level flows}
    \ForEach{$f \in F_{\text{core}}$}{
        Add $f$ to $P$\;
        $core\_links \gets core\_links - 1$\;
        Remove $f$ from $F_{\text{core}}$\;
        \If{$core\_links == 0$ \textbf{or} $F_{\text{core}} == \emptyset$}{
            \textbf{break}
        }
    }

    \tcp{Stage 2: Add aggregate-level flows}
    \ForEach{$f \in F_{\text{agg}}$}{
        Add $f$ to $P$\;
        $agg\_links \gets agg\_links - 1$\;
        Remove $f$ from $F_{\text{agg}}$\;
        \If{$agg\_links == 0$ \textbf{or} $F_{\text{agg}} == \emptyset$}{
            \textbf{break}
        }
    }

    Append $P$ to $P_{\text{list}}$\;
}

\Return $P_{\text{list}}$\;
\end{algorithm}

\subsection{Adaptive Routing}

Adaptive routing enables traffic to be split across multiple paths adaptively
based on the network condition, improving bandwidth utilization and often
reducing congestion. This is a very effective routing scheme for fat-tree.
However, in many scenarios where flow paths overlap significantly, causing
contention. On the other hand, for a full bisection bandwidth fat-tree, any
permutation traffic including the shift traffic pattern can be scheduled
without contention using single-path routing. Hence, for such traffic pattern,
using single path routing can achieve higher performance than
adaptive routing. Based on this observation, I propose \textit{SDN-adaptive}
that adapt between these two strategies: if the traffic demand $D$ is
less than a permutation, use SDN-optimal; otherwise, use the underlying
adaptive routing.

%I propose \textit{SDN-adaptive} routing for the multipath routing.
%\subsubsection{SDN-adaptive}
%The key idea behind SDN-adaptive is to first gather runtime information about the application’s traffic characteristics. During an initial profiling phase, SDN-adaptive collects communication performance data by running one iteration using adaptive multipath routing. This information is then sent to the SDN controller, which uses it decide the routing.
%The process begins by executing one iteration of the application using adaptive multipath routing. The SDN controller records the communication time from this iteration as a reference. Then, the full simulation is restarted using SDN-optimal routing. After completing the first iteration of this SDN-optimal run, the controller compares the current communication time with the previously recorded time from the adaptive run.
%
%If SDN-optimal demonstrates better performance (i.e., lower communication time), it is retained for the rest of the simulation. If, however, SDN-optimal is slower than adaptive, this indicates that multipath routing is more effective for the application's communication pattern. In that case, the controller immediately switches to adaptive routing for the remainder of the simulation.
%
%This design allows the routing policy to be tuned to the observed behavior of the application, ensuring better adaptability across diverse workloads. The algorithm is described in algorithm ~\ref{alg:sdn_adaptive_api}

%\begin{algorithm}[H]
%\DontPrintSemicolon
%\caption{SDN-adaptive algorithm}
%\label{alg:sdn_adaptive_api}
%
%\KwInput{%
%  Application $A$; \\
%  Topology $G$; \\
%  Routing schemes: SDN-optimal and Adaptive
%}
%\KwOutput{%
%  Final routing strategy and flow configuration
%}
%
%Run one iteration of $A$ using Adaptive routing\;
%Record $\mathrm{commTime}_{\text{Adaptive}}$\;
%
%Start full simulation of $A$ using SDN-optimal routing\;
%After first iteration, record $\mathrm{commTime}_{\text{SDN-optimal}}$\;
%
%\If{$\mathrm{commTime}_{\text{SDN-optimal}} < \mathrm{commTime}_{\text{Adaptive}}$}{
%  Continue simulation with SDN-optimal routing\;
%}
%\Else{
%  Switch to Adaptive routing for remainder of simulation\;
%}
%\end{algorithm}

