Eight representative applications are used in the study:
\texttt{Random-Permutation}, \texttt{Shift}, \texttt{Stencil3d}, \texttt{Stencil4d}, \texttt{Milc}, and \texttt{Nekbone}, \texttt{Random-Permutation-Mixed},
and \texttt{Stencil-Mixed}. All of the applications except \texttt{Random-Permutation-Mixed} and \texttt{Stencil-Mixed} including 
their computation and communication characteristics have been described
in detail in Chapter 2.

\begin{itemize}
\item \textbf{Random-Permutation-Mixed}: This pattern alternates between two distinct random-permutation communication phases, separated by a computation phase. Each communication phase uses a different source-destination mapping, introducing dynamic changes in the communication structure.

\item \textbf{Stencil-Mixed}: This application combines a \texttt{Stencil3d} phase followed by a \texttt{Stencil4d} phase, with an intermediate computation phase. The transition between patterns allows evaluation of the routing system’s responsiveness to changes in spatial communication demands.
\end{itemize}

All of these applications are communication-intensive, but 
they exhibit diverse communication behaviors ranging from synthetic
near-neighbor patterns to production-level scientific codes
and serve as appropriate test cases to assess how well SDN techniques
manage different types of network traffic.

To ensure accurate analysis of communication transitions, I enforce
explicit synchronization barriers between phases. This guarantees
that each communication phase is fully completed before the next begins,
preventing overlap and allowing isolated assessment of routing behavior across phases.
