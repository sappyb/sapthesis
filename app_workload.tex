Eight representative applications are used in the study:
\texttt{Random permutation}, \texttt{Shift}, \texttt{Stencil3d}, \texttt{Stencil4d}, \texttt{Milc}, and \texttt{Nekbone}, \texttt{Random permutation mixed}, \texttt{Stencil mixed}, and \texttt{Random permutation third}. All of the applications except \texttt{Random permutation mixed}, \texttt{Stencil mixed}, and \texttt{Random permutation third} have been described
in detail in Chapter 2.

\begin{itemize}
\item \textbf{Random permutation mixed}: This pattern alternates between two distinct random permutation communication phases, separated by a computation phase of 100ms. Each communication phase uses a different source-destination pair, introducing dynamic changes in the communication structure.

\item \textbf{Stencil mixed}: This application combines a \texttt{Stencil3d} phase followed by a \texttt{Stencil4d} phase, with an intermediate computation phase of 100ms. The transition between patterns allows evaluation of the routing system’s responsiveness to changes in spatial communication demands.

\item \textbf{Random permutation third}: In 3-to-1 tapered fat-tree there are a 3 flows which contends for one uplink. To have a complete no contention scenerio, we used this traffic pattern where we only selected one third of the flows that are originally present in a random-permuatation traffic.
\end{itemize}

To ensure accurate analysis of communication computation transitions in the Random permutation mixed and Stencil mixed, I enforce
explicit synchronization barriers between phases. This guarantees
that each communication phase is fully completed before the next begins,
preventing overlap and allowing isolated assessment of routing behavior across phases.

For the stencil applications, the problem dimensions are defined based on spatial decomposition across processing nodes. With 1024 nodes, the Stencil3D application is partitioned into a 3D grid of 8 nodes along the x-axis, 8 along the y-axis, and 16 along the z-axis. In the case of Stencil4D, the domain is decomposed into 8 nodes along the x-axis, 8 along the y-axis, 4 along the z-axis, and 4 along the w-axis, forming a four-dimensional near-neighbor structure. For 3-to-1 tapered fat-tree with 1536 nodes, the Stencil3D configuration extends to 8 × 8 × 24, while Stencil4D uses 8 × 8 × 6 × 4, maintaining the same spatial layout logic.

All of these applications are exhibit diverse communication behaviors ranging from synthetic
near-neighbor patterns to production-level scientific codes
and serve as appropriate test cases to assess how well SDN techniques
manage different types of network traffic.
