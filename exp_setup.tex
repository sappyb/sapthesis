The proposed techniques have been extensively studied using
the TraceR-CODES simulator~\cite{jain2016evaluating,mubarak2016enabling}.
TraceR-CODES is a software tool suite used for
performance analysis of parallel and distributed applications, and it
is specifically designed to simulate large-scale scientific applications
running on high-performance computing systems. In the following, I will first
discuss the experimental setup and the extensions that are added to
TraceR-CODES to support the evaluation. After that, the performance results
will be presented.

\subsection{Experimental Setup}

The experiments are performed on two distinct fat-tree configurations:
a 1024 node full bisection fat-tree and a 1536 node 3-to-1 tapered fat-tree
with a 3:1 tapering ratio. In both configurations, each switch
is equipped with 32 ports. All 32 ports of the core and aggregate switches are
fully utilized, connecting exclusively to other switches to preserve the
hierarchical structure of the topology.
The proposed SDN techniques for HPC environments including flow
identification, phase identification, and SDN routing are added to
TraceR-CODES. Flow identification schemes with and without user input
described in Section~\ref{sec:flow_identification}, including the
threshold-based scheme, the DNN-based scheme, user input based scheme
are incorporated in TraceR-CODES. 
To support the machine learning based flow identification scheme,
a trained DNN model has been integrated
into TraceR-CODES using Google TensorFlow's C APIs. To ensure
compatibility with the TensorFlow libraries, I utilized
MVAPICH 2 as the MPI implementation and adopted C++14 from GNU
version 9.0.1 as the compiler standard. The DNN model is able to 
process real-time network statistics and  
dynamically predicts network flows during the simulation runtime.
The user-input based scheme is supported by marking each MPI call with
user hints. 
The functionality to support phase identification with and without user information is added to the simulator, and the simulator can be configured to use a particular phase identification scheme. For dynamic phase identification, if the total data sent across the entire network is less than 100\,KB during a given interval, the network is classified as being in the computation phase. For routing, 
three types of SDN routing mechanisms are added: 
{\em SDN-greedy} that allocates paths by minimizing congestion in real time using simple heuristics, 
{\em SDN-optimal} that computes globally least-congested paths by evaluating all possible routes, and {\em SDN-adaptive} that dynamically adjusts
routing decisions based on the current traffic patterns and flow types.
Each of these methods is described in detail in earlier. 
To mitigate the impact of various types of delay on our data, specifically in 
the communication aspect of the experiment, we have configured the router 
delay, network interface controller (NIC) delay, software delay, and remote 
direct memory access (RDMA) delay to zero. This setup allows us to eliminate 
any extraneous delay factors and isolate the effects of the communication 
process on our data analysis.
The
Table ~\ref{tab:params.1} outlines the network configuration parameters used in the simulation.

\begin{table}[!htbp]
\centering
\caption{Network parameters for simulation of SHS}
\label{tab:params.1}
\vspace{1em}
\begin{tabular}{ll}
\toprule
Parameter & Value \\
\midrule
Packet Size     & 8192 Bytes \\
Switch Radix    & 32 \\
Link Bandwidth  & 11.9 GB/s \\
Eager Limit     & 64000 Bytes \\
NIC Scheduler   & Round-robin \\
\bottomrule
\end{tabular}
\end{table}
