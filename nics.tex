
The network interface controller (NIC) in contemporary HPC systems is
responsible for scheduling and packetizing messages.

\vspace{0.08in}
\noindent{\bf FCFS scheduling:} 
Messages are inserted
at the back of the scheduling queue, as and when they arrive.
During the packetization process, %the scheduler pops the elements from the head of
%the scheduling queue, creates the packet from that message, and pushes it back
%to the top.
the scheduler keeps creating packets from the top of the queue until
the entire message is packetized before it packetizes the next message in the queue. 
%This is what typically happens when there is
%only one virtual channel in a NIC/network.

\vspace{0.08in}
\noindent{\bf RR scheduling:} Messages are inserted into the scheduling queue of
the network interface, as and when they arrive. During the packetization process,
the scheduler creates one packet for a message and then moves to the next message:
all messages are considered in a round-robin manner. 
%However during the packetization process, the scheduler pops the top message from
%scheduling queue, creates a packet from it and then inserts to the tail. The scheduler
%then moves to the next message in the queue, which is currently at the top of the
%scheduling queue. The scheduler repeats the process until all the messages in the
%scheduling queue are packetized before it again creates a packet from the first message
%which entered the queue initially.
RR not only allows concurrent communication
progress for several communicating-pairs, but may also help the network in
better utilizing multiple communication paths. While desirable, such a scheme is
difficult to implement in the hardware as the number of concurrent messages can
be very large.

%default first come first serve (FCFS) message
%scheduling. 
\vspace{0.08in}
%In RR-N, the messages are entering the scheduling queue in the network interface
%in a first come first serve manner with the first N message in the queue, waiting
%for their turn of packetization. Each of those N messages is first packetized one
%after the other. When one of the N messages is completely packetized, the immediate
%next message in the queue takes its place. 

\vspace{0.08in}
\noindent{\bf RR-N scheduling:} In this scheme, $N$ is a parameter. RR-N is similar to RR,
except that instead of packetizing every message in the scheduling queue in
a round-robin manner,
the scheduler packetizes the top $N$ messages in the scheduling queue. For example,
in RR-2, the scheduler only packetizes the first 2 messages for communication.
This newly added scheme simulates the real world scenario where a limited
number of hardware queues are available at a NIC, which are used to keep
multiple message in-flight concurrently.
%and thus
%packetization of only these many messages can happen concurrently.  

