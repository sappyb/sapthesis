\chapter{Conclusion}
In my dissertation research, I focus on two critical areas in High-Performance Computing (HPC): optimizing hardware parameters for next-generation GPU-based platforms and integrating Software Defined Networking (SDN) to enhance HPC application performance.

In the first part, I explore optimizing hardware parameters for GPU-based HPC platforms. With the current trend of HPC systems moving toward higher computational capacity GPU nodes, it is essential to evaluate the impact of key hardware design parameters—such as the number of GPUs per node, network link bandwidth, and network interface controller (NIC) scheduling policies—within fat-tree and dragonfly topologies. Using the TraceR-CODES simulation tool, I analyze the effects of these parameters on the computation and communication capacities for various HPC applications. The results indicate that as more GPUs are integrated per node, the sensitivity of applications to communication performance increases, necessitating higher network bandwidth and effective scheduling methods to maintain optimal system performance. The exact impact of these hardware parameters is application-dependent, highlighting the need for tailored investigations to determine cost-effective configurations.

In the second part, I investigate routing optimization strategies for HPC networks using software-defined networking (SDN). Specifically, I develop a suite of SDN-based routing algorithms—adaptive, optimal, and a hybrid scheme tailored for tapered fat-tree topologies—to address performance limitations caused by flow-level contention. The SDN-optimal algorithm ensures contention-free routing in full-bisection fat-trees by leveraging a graph-theoretic formulation and Hall’s Marriage Theorem. For tapered fat-trees, where upper-layer bandwidth is reduced, I introduce a permutation-splitting strategy to construct sub-permutations that maintain link-level contention bounds. Additionally, I propose an SDN-adaptive approach that selects between adaptive and optimal routing modes dynamically based on runtime flow behavior. Simulation results demonstrate that these routing strategies reduce congestion and improve communication performance in an application-dependent manner, highlighting the importance of topology-aware, programmable routing in next-generation HPC systems.

In summary, my dissertation research contributes to optimizing HPC platforms by addressing both hardware and networking challenges. By enhancing GPU integration and utilizing SDN for better network management, I provide practical solutions for developing next-generation HPC systems that achieve optimal performance and efficiency.
