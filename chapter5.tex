\chapter{Conclusion}
In my dissertation research, I focus on two critical areas in HPC: integrating SDN techniques to enhance HPC application performance and optimizing hardware parameters for next-generation GPU-based platforms.

In the first part of this study, I articulate an SDN-based HPC system that integrates flow identification, phase detection, and SDN routing to mitigate contention in contemporary fat-tree interconnects. I have devised two complementary flow-classification techniques: (i) a lightweight, deep-learning-based detector that recognizes elephant flows within a single SDN polling interval, and (ii) an application-level API through which HPC codes can label their own flows—an approach that eliminates the multi-second delays typical of data-centre polling. I have paired these classifiers with both explicit user phase markers and a dynamic detector so that network resources are released the instant an application transitions between communication and computation phases. With accurate, phase-aware demand profiles in hand, I have implemented three routing schemes. SDN-greedy spreads traffic by minimising the worst-link load; SDN-optimal leverages a graph-theoretic edge-colouring to deliver contention-free schedules on full-bisection fat-trees; and I have modified SDN-optimal to operate on 3-to-1 tapered fat-trees and introduce SDN-adaptive, a multipath solution for scenarios in which traditional adaptive routing underperforms. Simulation results demonstrate that these routing strategies reduce congestion and improve communication performance in an application-dependent manner, highlighting the importance of SDN-based techniques such as programmable routing in next-generation HPC systems.

In the second part, I explore optimizing hardware parameters for GPU-based HPC platforms. With the current trend of HPC systems moving toward higher computational capacity GPU nodes, it is essential to evaluate the impact of key hardware design parameters—such as the number of GPUs per node, network link bandwidth, and network interface controller (NIC) scheduling policies—within fat-tree and dragonfly topologies. Using the TraceR-CODES simulation tool, I have analyzed the effects of these parameters on the computation and communication capacities for various HPC applications. The results indicate that as more GPUs are integrated per node, the sensitivity of applications to communication performance increases, necessitating higher network bandwidth and effective scheduling methods to maintain optimal system performance. The exact impact of these hardware parameters is application-dependent, highlighting the need for tailored investigations to determine cost-effective configurations.

In summary, my dissertation research contributes to optimizing HPC platforms by addressing both networking and hardware challenges. By utilizing SDN and enhancing GPU integration for better network management, I provide practical solutions for developing next-generation HPC systems that achieve optimal performance and efficiency.
